% Created 2020-09-03 Thu 15:53
% Intended LaTeX compiler: pdflatex
\documentclass[11pt]{article}
\usepackage[utf8]{inputenc}
\usepackage[T1]{fontenc}
\usepackage{graphicx}
\usepackage{grffile}
\usepackage{longtable}
\usepackage{wrapfig}
\usepackage{rotating}
\usepackage[normalem]{ulem}
\usepackage{amsmath}
\usepackage{textcomp}
\usepackage{amssymb}
\usepackage{capt-of}
\usepackage{hyperref}
\usepackage{minted}
\IfFileExists{./resources/style.sty}{\usepackage{resources/style}}{}
\IfFileExists{./resources/referencing.sty}{\usepackage{resources/referencing}}{}
\addbibresource{resources/references.bib}
\author{Ryan Greenup}
\date{\today}
\title{Data Sci Discover Project}
\hypersetup{
 pdfauthor={Ryan Greenup},
 pdftitle={Data Sci Discover Project},
 pdfkeywords={},
 pdfsubject={},
 pdfcreator={Emacs 27.1 (Org mode 9.4)}, 
 pdflang={English}}
\begin{document}

\maketitle
\tableofcontents

\section{Simplifying Power Walk to be solved with Sparse Matrices}
\label{sec:orgb80aad4}
The Random Surfer model is:

$$\begin{aligned}
    \mathbf{S} &= \alpha \mathbf{T} +  \mathbf{F}  \label{eq:sparse-RS}\end{aligned}$$

where:

\begin{itemize}
\item \(\mathbf{T}\)

\begin{itemize}
\item is an \(i \times j\) matrix that describes the probability of
travelling from vertex \(j\) to \(i\)

\begin{itemize}
\item This is transpose from the way that \texttt{igraph} produces an adjacency
matrix.
\end{itemize}
\end{itemize}

\item \(\mathbf{F} = \begin{bmatrix} \frac{1}{n} \\ \frac{1}{n} \\ \frac{1}{n} \vdots \end{bmatrix}\)
\end{itemize}

Interpreting the transition probability matrix in this way is such that
\(\mathbf{T}= \mathbf{A}\mathbf{D}^{- 1}_A\) under the following
conditions:

\begin{itemize}
\item No column of \(\mathbf{A}\) sums to zero

\begin{itemize}
\item If this does happen the question arises how to deal with
\(\mathbf{D_\mathbf{A}^{- 1}}\)

\begin{itemize}
\item I've been doing
\(D'_{\mathbf{A}, i, j} := \mathtt{diag} \left( {\frac{1}{\mathtt{colsums}\left( \mathbf{A} \right)}} \right)\)
and then replacing any \(0\) on the diagonal with 1.
\end{itemize}

\item What is done in the paper is to make another matrix \(\mathbf{Z}\)
that is filled with 0, if a column sum of \(\mathbf{A}\) adds to zero
then that column in \(\mathbf{Z}\) becomes \(\frac{1}{n}\)

\begin{itemize}
\item This has the effect of making each row identical

\item The probability of going from an orphaned vertex to any other
vertex would hence be \(\frac{1}{n}\)

\item The idea with this method is then to use
\(D_\mathbf{\left( A+Z \right)}^{- 1}\) this will be consistent with
the \emph{Random Surfer} the method using \(\mathbf{F}\) in
]
\end{itemize}

where each row is identical that is a 0
\end{itemize}
\end{itemize}

The way to deal with the \emph{Power Walk} is more or less the same.

observe that:

$$\begin{aligned}
    \mathbf{B} = \beta^{\mathbf{A}} \wedge \mathbf{A}_{i, j}\in \mathbb{R}  \implies  \left\lvert \mathbf{B}_{i, j} \right\rvert > 0 \quad \forall i,j>n\in \mathbb{Z}^+ \label{eq:b-is-pos}\end{aligned}$$

Be mindful that the use of exponentiation in
] is not an element wise exponentiation
and not an actual matrix exponential (which would be defined by using
power series and logs but is defined)

So if I have:

\begin{itemize}
\item \(\mathbf{O}_{i, j} := 0, \quad \forall i,j\leq n \in \mathbb{Z}^+\)

\item \(\vec{p_i}\) as the state distribution, being a vector of length \(n\)
\end{itemize}

Then It can be shown (see \eqref{eq:sparse-power-walk}):

$$\begin{aligned}
    \mathbf{O} \mathbf{D}_{\mathbf{B}}^{-1} \vec{p_i} = \vec{\delta} ^{\mathbf{T}}\vec{p_i} \label{eq:pw-delta-simp}\end{aligned}$$

where:

\begin{itemize}
\item \(\vec{\delta_i} = \frac{1}{\mathtt{colsums} \left( \mathbf{B} \right)}\)
\end{itemize}

This means we can do:

$$\begin{aligned}
     \vec{p_{i +  1}} &=  \mathbf{B} \mathbf{D}_{\mathbf{B}}^{- 1} \vec{p_{i}}  \\
     &= \left( \mathbf{B} -  \mathbf{O} +  \mathbf{O} \right) \mathbf{D}_{\mathbf{B}}^{- 1}\vec{p_i} \\
     &= \left( \left( \mathbf{B} -  \mathbf{O} \right) \mathbf{D}_{\mathbf{B}}^{- 1} +  \mathbf{O}\mathbf{D}_{\mathbf{B}}^{- 1} \right) \vec{p_i} \\
     &= \left( \mathbf{B}-  \mathbf{O}\right) \mathbf{D}_{\mathbf{B}}^{- 1} \vec{p_i} +  \mathbf{O} \mathbf{D}_{\mathbf{B}}^{- 1} \vec{p_i}  \\
     &= \left( \mathbf{B}- \mathbf{O} \right)\mathbf{D}_{\mathbf{B}}^{- 1} \vec{p_i} +  \vec{\delta'}\vec{p_i} \vec{1}
 \end{aligned}$$






\begin{align}
  \mathbf{OD}_{\mathbf{B}}^{- 1} \overrightarrow{p_i} & = \left(
  \begin{array}{cccc}
    1 & 1 & 1 & \\
    1 & 1 & 1 & \ldots\\
    1 & 1 & 1 & \\
    & \vdots &  & \ddots
  \end{array} \right) \left(\begin{array}{cccc}
    \sum^n_{i = 1} [\ensuremath{\boldsymbol{A}}_{i, 1}] & 0 & 0 & \\
    0 & \sum^n_{i = 1} [\ensuremath{\boldsymbol{A}}_{i, 2}] & 0 & \ldots\\
    0 & 0 & \sum^n_{i = 1} [\ensuremath{\boldsymbol{A}}_{i, 3}] & \\
    0 & \vdots & 0 & \ddots
  \end{array}\right) \left(\begin{array}{c}
    p_1\\
    p_2\\
    p_3\\
    \vdots
  \end{array}\right) \nonumber \\
  & =  \left(\begin{array}{cccc}
    \frac{p_1}{\sum^n_{i = 1} [\ensuremath{\boldsymbol{A}}_{i, 1}]} & \frac{p_2}{\sum^n_{i =
    1} [\ensuremath{\boldsymbol{A}}_{2, 1}]} & \frac{p_3}{\sum^n_{i = 1} [\ensuremath{\boldsymbol{A}}_{3,
    1}]} & \\
    \frac{p_1}{\sum^n_{i = 1} [\ensuremath{\boldsymbol{A}}_{i, 1}]} & \frac{p_2}{\sum^n_{i =
    1} [\ensuremath{\boldsymbol{A}}_{2, 1}]} & \frac{p_3}{\sum^n_{i = 1} [\ensuremath{\boldsymbol{A}}_{3,
    1}]} & \ldots\\
    \frac{p_1}{\sum^n_{i = 1} [\ensuremath{\boldsymbol{A}}_{i, 1}]} & \frac{p_2}{\sum^n_{i =
    1} [\ensuremath{\boldsymbol{A}}_{2, 1}]} & \frac{p_3}{\sum^n_{i = 1} [\ensuremath{\boldsymbol{A}}_{3,
    1}]} & \\
    \vdots & \vdots & \vdots & \vdots
  \end{array}\right) \nonumber \\
  & =  \left(\begin{array}{ccc}
    \frac{1}{\sum^n_{i = 1} [\ensuremath{\boldsymbol{A}}_{i, 1}]} & \frac{1}{\sum^n_{i = 1}
    [\ensuremath{\boldsymbol{A}}_{i, 2}]} & \frac{1}{\sum^n_{i = 1} [\ensuremath{\boldsymbol{A}}_{i, 3}]}
  \end{array}\right)  \left(\begin{array}{c}
    p_1\\
    p_2\\
    p_3
  \end{array}\right) \left(\begin{array}{c}
    1\\
    1\\
    1
  \end{array}\right)\\
  & = \ensuremath{\boldsymbol{\delta}}^{\ensuremath{\boldsymbol{T}}} \overrightarrow{p} \label{eq:sparse-power-walk}
  \overrightarrow{1}\end{align}
\end{document}
