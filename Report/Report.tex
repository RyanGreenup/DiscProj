% Created 2020-10-11 Sun 22:23
% Intended LaTeX compiler: pdflatex
\documentclass[11pt]{article}
\usepackage[utf8]{inputenc}
\usepackage[T1]{fontenc}
\usepackage{graphicx}
\usepackage{grffile}
\usepackage{longtable}
\usepackage{wrapfig}
\usepackage{rotating}
\usepackage[normalem]{ulem}
\usepackage{amsmath}
\usepackage{textcomp}
\usepackage{amssymb}
\usepackage{capt-of}
\usepackage{hyperref}
\usepackage{minted}
\IfFileExists{./resources/style.sty}{\usepackage{./resources/style}}{}
\IfFileExists{./resources/referencing.sty}{\usepackage{./resources/referencing}}{}
\addbibresource{./resources/references.bib}
\usepackage[mode=buildnew]{standalone}
\usepackage{tikz}
\usetikzlibrary{decorations.fractals}
\usetikzlibrary{lindenmayersystems}
\author{Ryan Greenup}
\date{\today}
\title{Page Rank}
\hypersetup{
 pdfauthor={Ryan Greenup},
 pdftitle={Page Rank},
 pdfkeywords={},
 pdfsubject={},
 pdfcreator={Emacs 27.1 (Org mode 9.4)}, 
 pdflang={English}}
\begin{document}

\maketitle
\tableofcontents


\subsection{Introduction}
\label{sec:org140cc19}
The centrality score of a graph is a metric that measures the
importance and popularity of a vertex. \footnote{For a small graph drawn
in a way to minimise overlapping edges the centremost geometric vertex
will coincide with the highest centrality score, for example in figure
\ref{irreducible-example} vertex \(D\) is the vertex with the highest
frequency during a random walk}


The \emph{PageRank} method asserts that the centrality of a vertex can be
measured by the frequency of incidence with that vertex during a
random walk.

\subsection{Implementing PageRank Generally}
\label{PageRank-Generally}
A graph can be expressed as an adjacency matrix \(\mathbf{A}\):

\[
A_{i,j} \in \left\{ 0,1 \right\}
\]

Where each element of the matrix indicates whether or not travel from
vertex \(j\) to vertex \(i\) is possible with a value of 1. \footnote{Some
authors define an adjacency matrix transposed (see e.g.
\cite{rosenDiscreteMathematicsIts2007,AdjacencyMatrix2020a,meghabghabSearchEnginesLink2008})
this unfourtunately includes the \texttt{igraph} library
\cite{gaborcsardiIgraphManualPages2019} but that convention will not be
followed in this paper}

During a random walk the probability of arriving at vertex \(j\) from vertex
\(i\) can similarly be described as an element of a transition probability
matrix \(\mathbf{T}_{i,j}\), this matrix can be described by the following
relationship \footnote{In this paper \(\vec{1}\) refers to a vector containing only
values of 1, the size of which should be clear from the context}:

\begin{align}
\mathbf{T} &= \mathbf{A} \mathbf{D}^{-1}_{\mathbf{A}} \label{eq:basic-trans-def} : \\
& \mathbf{D}_{\mathbf{A}} = \mathrm{diag}\left(\vec{1} \mathbf{A}\right) \label{eq:diagScaleDef}
\end{align}

The value of \(\mathbf{D}\) is such that under matrix multiplication
\(\mathbf{A}\) will have columns that sum to 1 (i.e. a \emph{column
stochastic matrix}, see \textsection \ref{definitions}), for a reducible
or non-stochastic graph the definition of \(\mathbf{D}\) would need to
be adjusted to acheive this, this is discussed below 

During the random walk, the running tally of frequencies, at the
\(i^{\mathrm{th}}\) step of the walk, can be described by a vector
\(\vec{p}\), this vector can be determined for each step by matrix
multiplication:

\begin{align}
\vec{p_{i+1}} = \mathbf{T}\vec{p_{i}} \label{eq:recurrence}
\end{align}

This relationship is a linear recurrence relation, more importantly
however it is a \emph{Markov Chain}
\cite[\textsection 4.4]{langvilleGooglePageRankScience2012}.

Finding the Stationary point for this relationship will give a
frequency distribution for the nodes and a metric to measure the
centrality of vertices.

\subsection{Definitions}
\label{definitions}
The following definitions are used in this report \footnote{see generally \cite[Ch. 15]{langvilleGooglePageRankScience2012} for further reading} :

\begin{description}
\item[{Markov Chains}] are discrete mathematical model such that future values depend only on current values \cite[\textsection 1.5]{foussAlgorithmsModelsNetwork2016}
\item[{Stochastic Matrices}] contain only positive values where each column sums to 1 \cite{langvilleGooglePageRankScience2012,larsonElementaryLinearAlgebra1991}
\begin{itemize}
\item some authors use rows (see e.g. \cite[\textsection 15.3]{langvilleGooglePageRankScience2012}), in this paper columns will be used, i.e. columns will add to one and an entry \(\mathbf{A}_{i,j} \neq 0\) will indicate that travel is permitted from vertex \(j\) to vertex \(i\).
\begin{itemize}
\item \emph{Column Stochastic} and \emph{Row Stochastic} can be used to more clearly distinguish between which type of stochastic matrix is being used.
\end{itemize}
\item Many programming languages return \emph{unit-eigenvectors} \(\vec{x}\) such that \(\left\lvert \left\lvert \vec{x} \right\rvert \right\rvert = 1\) as opposed to \(\mathtt{sum} \left( \vec{\mathtt{x}}\right) = 1\), so when solving for a stationary vector it can be necessary to perform \(\vec{\mathtt{p}} \leftarrow \frac{\vec{\mathtt{p}}}{\sum \vec{p}}\)
\end{itemize}
\item[{Irreducible}] graphs have a path from from any given vertex to another vertex. \cite[\textsection 15.2]{langvilleGooglePageRankScience2012}
\begin{description}
\item[{Ergodic}] graphs are irreducible graphs with further constraints outside
the scope of this report (see e.g.
\cite{nathanaelackermancameronfreeralexkruckmanandrehanapatelProperlyErodicStructures2017,chenEigenvaluesInequalitiesErgodic2005})
\begin{itemize}
\item It is a necessary but not a sufficient condition of ergodic graphs that all vertices be reachable from any other vertices (see \cite{sazProbabilityTheoryThis} for a counter example.)
\end{itemize}
\end{description}
\item[{Primitive Matrices}] are non-negative irreducible matrices that have only one eigenvalue on the unit circle.
\begin{itemize}
\item If a matrix is primitive it will approach a limit under exponentiation \cite[\textsection 15.2]{langvilleGooglePageRankScience2012}
\end{itemize}
\item[{Transition Probability Matrix}] is a stochastic matrix where each column is a vector of probabilities such that \(\mathbf{T}_{i,j}\) represents the probability of travelling from vertex \(j\) to vertex \(i\) during a random walk.
\begin{itemize}
\item Some Authors consider the transpose (see e.g. \cite{langvilleGooglePageRankScience2012}).
\end{itemize}
\item[{Aperiodic}] Markov chains are markov chains with an irreducible and primitive transition probability matrix.
\begin{itemize}
\item If the transition probability matrix is irreducible and imprimitive it is said to be a periodic Markov chain.
\end{itemize}
\item[{Regular}] Markov Chains are regular irreducible and aperiodic.
\item[{Sparse}] Matrices contain a majority of elements with values equal to 0 \cite[\textsection 4.2]{langvilleGooglePageRankScience2012}
\item[{Sparse}] Iterating the
\item[{PageRank}] A process of measuring graph centrality by using a random walk algorithm and measuring the most frequent node
\begin{itemize}
\item In the literature (see e.g. \cite{guptaWTFWhoFollow2013,langvilleGooglePageRankScience2012}) the Random Surfer model is usually used to refer to the introduction of a probability of travelling to any other node, this is discussed in CROSSREF
\end{itemize}
\end{description}

\subsubsection{Notation}
\label{notation}
\begin{description}
\item[{\(\mathbf{A}\)}] Is the adjacency matrix of a graph
\begin{description}
\item[{\(\mathbf{A}_{i,j} = 1\)}] Indicates that \(j\) and \(i\) are adjacent vertices.
\end{description}
\item[{\(\mathbf{A}\left[:,j\right]\)}] Refers to the \(j^{\mathrm{th}}\) column vector of \(\mathrm{A}\)
\begin{itemize}
\item This syntax is much like \emph{Julia} or \emph{Python} but also occurs in the literature, see e.g. \cite[\textsection 1.1.8]{golubMatrixComputations1996}
\end{itemize}
\item[{\(\mathbf{T}\)}] Is the transition probability matrix of a graph
\begin{itemize}
\item \(\mathf{T}_{i,j}\) is equal to the probability of travelling \(j \rightarrow  i\) during a random walk.
\begin{itemize}
\item \(\mathbf{T} = \mathbf{A} \mathbf{D}_{\mathbf{A}}^{-1}\)
\begin{itemize}
\item Where \(\mathbf{D}^{-1}\) is a matrix such that multiplication with which scales each column of \(\mathbf{A}\) to 1.
\begin{itemize}
\item \(\mathbf{D}^{-1}_{\mathbf{A}} = \vec{1}\mathbf{D}^{-1}_{\mathbf{A}} = \frac{1}{\vec{1}\mathbf{D}_{\mathbf{A}}}\) for some stochastic matrix \(\mathbf{A}\)
\end{itemize}
\end{itemize}
\end{itemize}
\end{itemize}
\item[{\(n\)}] Refers to the number of vertices in a graph elements of a matrix
\begin{itemize}
\item \(n = \mathtt{nrow}\left(\mathbf{A}\right) = \mathtt{ncol}\left(\mathbf{A}\right)\)
\end{itemize}
\item[{\(\mathbf{B}_{i,j} = \frac{1}{n}\)}] Is a matrix of size \(n\times n\) representing the background probability of uniformly selecting any vertex of a graph.
\item[{\(\vec{1}\)}] is a vector of length \(n\) containing only the value 1.
\begin{itemize}
\item The convention that a vector behaves as a vertical \(n \times 1\) matrix will be used here.
\item Some authors use \(\mathbf{e}\), see e.g. \cite{langvilleGooglePageRankScience2012}
\end{itemize}
\item[{\(\mathbf{J} = \vec{1}\cdot \vec{1}^{\mathrm{T}} \iff \mathbf{J}_{i,j} = 1\)}] Is a completely dense \(n \times n\) matrix.
\begin{itemize}
\item It's worth noting that \(\mathbf{E}, \mathbf{J}\) are common choices for this matrix.
\end{itemize}
\item[{\(\alpha\)}] The probability of teleporting from one vertex to another during a random walk.
\begin{itemize}
\item In the literature \(\alpha\) is often referred to as a damping factor (see e.g.  \cite{berkhoutRankingNodesGeneral2018a,brinkmeierPageRankRevisited2006a,fuDampingFactorGoogle2006,kamvarAdaptiveMethodsComputation2004b,bianchiniPageRank2005})
\end{itemize}
or a smoothing constant (see e.g \cite{koppelMeasuringDirectIndirect2014}).
\item \(\vec{p}_{i} = \frac{\mathrm{deg}(v_{1})}{\mathrm{vol}(G)}\)
\begin{itemize}
\item \(\mathrm{vol}(G) = \sum^{n}_{i = 1} \left[ \mathrm{indeg}(v) \right] = \sum^{n}_{i = 1} \left[ \mathrm{outdeg}(v) \right ] = \sum^{n}_{i = 1} \left[ \mathrm{deg}(v) \right]\)
\end{itemize}
\end{description}

\subsection{Random Surfer Model}
\label{sec:orga18bc07}
\subsubsection{Issues}
\label{issues}
The approach in \ref{PageRank-Generally} has the following issues

\begin{enumerate}
\item Convergence of \eqref{eq:recurrence}
\begin{enumerate}
\item Will this relationship converge or diverge?
\item How quickly will it converge?
\item Will it converge uniquely?
\end{enumerate}
\item Reducible graphs
\begin{enumerate}
\item If it is not possible to perform a random walk across an entire graph for all initial conditions, this approach doesn't have a clear analogue.
\end{enumerate}
\item Cycles
\begin{enumerate}
\item A graph that is cyclical may not converge uniquely
\begin{enumerate}
\item Consider for example the graph \(A\rightarrow B\).
\end{enumerate}
\end{enumerate}
\end{enumerate}

\subsubsection{Markov Chains}
\label{markov}
The relationship in \eqref{eq:recurrence} is a \emph{Markov Chain}  and it is known
that the power method will converge: \footnote{A \emph{Markov Chain} is
simply any process that evolves depending on it's current condition, it's
interesting to note however that the theory of \emph{Markov Chains} is not mentioned in any
of the original papers by page and brin
\cite[\textsection 4.4]{langvilleGooglePageRankScience2012}}

\begin{itemize}
\item for a stochastic irreducible markov chain \cite[\textsection 1.5.5]{larsonElementaryLinearAlgebra1991},
\item regardless of the initial condition of the process for an \emph{aperiodic} Markov chain \cite[\textsection 4.4]{langvilleGooglePageRankScience2012}
\end{itemize}

\paragraph{Stochastic}
\label{stochastic}
If a vertex had a 0 outdegree the corresponding column sum for the adjacency
matrix describing that graph would also be zero and the matrix non-stochastic,
this could occur in the context of a random walk where a link to a page with no
outgoing links was followed (e.g. an image), this would be the end of the
walk.

So to ensure that \eqref{eq:recurrence} will converge, the probability transition
matrix must be made stochastic, to acheive this a uniform probability of teleporing from a dead end to any other vertex can be introduced:

\begin{align}
\mathrm{S} = \mathrm{T}+ \frac{\vec{a} \cdot \vec{1}^{\mathrm{T}} }{n} \label{eq:nearly-random-surfer}
\end{align}

This however would not be sufficient to ensure that \eqref{eq:recurrence} would converge, in addition the transition probability matrix must be made irreducible and aperiodic (i.e. primitive). \cite{langvilleGooglePageRankScience2012}

\begin{figure}[htbp]
\centering
\includegraphics[width=6cm]{media/dot/stochastic_graph_example.dot.png}
\caption{\label{fig:stochastic-example}\(D\) is a \emph{dangling node}, a dead end during a random walk, the corresponding probability transition matrix \((\mathbf{T})\) is hence non-stochastic (and also reducible), Introducing some probability of teleporting from a dead end to any other vertex as per \eqref{eq:nearly-random-surfer} (denoted in red) will cause \(\mathbf{T}\) to be stochastic.}
\end{figure}

\paragraph{Irreducible}
\label{sec:orgc7c0600}
A graph that allows travel from any given vertex to any other vertex is said to be irreducible \cite{langvilleGooglePageRankScience2012}, see for example figure \ref{irreducible-example}, this is important in the context of a random walk because only in an irreducible graph can all vertexes be reached from any initial condition.

\begin{figure}[htbp]
\centering
\includegraphics[width=6cm]{media/dot/reducible_graph_example.dot.png}
\caption{\label{irreducible-example}Example of a reducible graph, observe that although \(C\) is not a dead end as discussed in \ref{stochastic}, there is no way to travel from \(C\) to \(A\), by adding an edge such an edge in the resulting graph is irreducible. The resulting graph is also aperiodic (due to the loop on \(B\)) and stochastic, so there will be a stationary distribution corresponding to \eqref{eq:recurrence}.}
\end{figure}

\paragraph{Aperiodic}
\label{sec:org5bcd84b}
An a periodic graph has only one eigenvalue that lies on the unit circle, this is important because \(\lim_{k\rightarrow \infty} \left( \frac{\mathbf{A}}{r}^{k} \right)\) exists for a non-negative irreducible matrix \(\mathbf{A}\) if and only if \mathbf{A} is aperiodic. A graph that is a periodic can be made aperiodic by interlinking nodes \footnote{Actually it would be sufficient to merely link one vertex to itself \cite[\textsection 15.2]{langvilleGooglePageRankScience2012} but this isn't very illustrative or helpful in this context}


\begin{figure}[htbp]
\centering
\includegraphics[width=6cm]{media/dot/aperiodic.dot.png}
\caption{\label{fig:aperiodic}A periodic graph with all eigenvalues on the unit circle \(\xi = \frac{\sqrt{2}}{2} e^{\frac{\pi i}{4} k}\), by adding in extra edges the graph is now aperiodic, this does not represent the random surfer moidel, which would in theory connect every vertex but with some probability.}
\end{figure}

\paragraph{The Fix}
\label{fix}
To ensure that the transition probability matrix is primitive (i.e. irreducible and aperiodic) as well as stochastic, instead of introducing the possible to teleport out of dead ends, introduce a probability of teleporting to any node at any time (\(\alpha\)), this approach is known as the \emph{Random Surfer} model and the transition probability matrix is given by \cite{larrypageAnatomyLargescaleHypertextual1998} :

\begin{align}
\mathbf{S} = \alpha \mathbf{T} + \frac{(1- \alpha)}{n} \mathbf{J} \label{eq:random-surfer}
\end{align}

This matrix is primitive and stochastic and so will converge (it is also unfourtunately completely dense, see \ref{solving-stationary-dist} \cite[\textsection 4.5]{langvilleGooglePageRankScience2012}.

The relation ship in \eqref{eq:recurrence} can now be re expressed as:

\begin{align}
\vec{p_{i+1}} \rightarrow \mathbf{T} \vec{p}_{i} \label{eq:random-surfer-recurrence}
\end{align}



\begin{figure}[htbp]
\centering
\includegraphics[width=6cm]{media/dot/random_surfer.dot.png}
\caption{\label{fig:rseg}A graph that is aperiodic, reducible and non-stochastic, by applying the random surfer model \eqref{eq:random-surfer} blue \emph{teleportation} edges are introduced, these may be followed with a probability of \(1 - \alpha\)}
\end{figure}
\subsubsection{Limitations}
\label{sec:org2f22083}
The \emph{Random Surfer} Model can only consider positively weighted edges, it cannot
take into account negatively weighted edges. This limitation is increasingly
important as techniques of sentiment analysis are developed which could indicate
that links promote aversion rather than endorsement (e.g. a negative review or
an innapropriate advertisement).
\subsection{Power walk}
\label{pwalk}
The \emph{Power Walk} method is an alternative approach to develop a probability
transition matrix to use in place of \eqref{eq:recurrence}.

Let the probability of travelling to a non-adjacent vertex be some value \(x\)
and \(\beta\) be the ratio of probability between following an edge or
teleporting to another vertex.

This transition probability matrix would be such that the probability of
travelling some vertex \(j \rightarrow i\) would be :

\begin{align}
\mathbf{W}_{i, j} = x\beta^{\mathbf{A_{i,j}}} \label{eq:prob-power-walk}
\end{align}

Where \(\mathbf{W}\) denotes the power walk probability transition matrix.

Whe probability of travelling to any given vertex must be 1 and so:


\begin{align}
      1 &= \sum^{n}_{j= 1}   \left[ x \beta^{\mathbf{A_{i,j}}} \right] \\
       \implies  x&= \left( \sum^{n}_{j= 1}   \beta^{\mathbf{A_{i,j}}}
       \right)^{-1} \label{eq:powerwalk-x-val}
\end{align}

Substituting the value of \(x\) from \eqref{eq:powerwalk-x-val} into \eqref{prob-power-walk} gives the probability as:

\begin{align}
      \mathbf{W}_{i,j} &= \frac{\beta^{\mathbf{A}__i,j}}{\sum^{n}_{i=j}
      \left[ \beta^{\mathbf{A}_{i,j}} \right] }
\end{align}

In this model all vertices are interconnected by some probability of jumping to
another vertex, so much like the random surfer model \eqref{eq:random-surfer} discussed
at \ref{fix} \(\mathbf{W}\) will be a primitive stochastic matrix and so if
\(\mathbf{W}\) was used in place of \(\mathbf{T}\) in \eqref{eq:recurrence} a solution
would exist.

\subsection{Sparse Matrices}
\label{sec:org35ab689}
Most Adjacency matrices resulting from webpages and analagous networks
result in sparse adjacency matrices (see figure \ref{fig:den_undir_ba}),
this is a good thing because it requires far less computational
resources to work with a sparse matrix than a dense matrix
 \cite[\textsection 4.2]{langvilleGooglePageRankScience2012} .

Sparse matrices can be expressed in alternetive forms so as to reduce the memory
footprint associated with that matrix, one such method is the \emph{Compressed Row Storage} method, this involves listing the elements as a table as in \eqref{eq:ordinary} and \eqref{eq:crc}.

This is implemented in \textbf{\emph{R}} with the \texttt{Matrix} package
\cite{batesMatrixSparseDense2019a} .

\begin{align}
    \begin{bmatrix}
	1 & 0 & 0 & 0 & 0 \\
	0 & 0 & 0 & 0 & 0 \\
	0 & \phi & 0 & 0 & 0 \\
	0 & 0 & 0 & 0 & \pi \\
	0 & 0 & 0 & 0 & 0 \\
    \end{bmatrix}  \label{eq:ordinary} \\
    \ \nonumber \\
    \ \nonumber \\
    \begin{matrix}
	\mathrm{Row\ Index} & \mathrm{Col\ Index} & \mathrm{Value}\\
	1 & 1 & 1 \\
	3 & 2 & \phi \\
	4 & 5 & \pi \\
    \end{matrix}  \label{eq:crc}
\end{align}


\subsubsection{Solving the Stationary Distribution}
\label{solving-stationary-dist}
The relationship in \eqref{eq:recurrence} \footnote{This assumes that the transition
probability matrix is stochastic and primitive as it would be for \(\mathbf{S}\)
and \(\mathbf{W}\)} is equivelant to the eigenvalue value problem, where
\(\vec{p} = \lim_{i \rightarrow \infty} \left( \vec{p_{i}}\right)\) is the
eigenvector \footnote{More accurately the eigenvector specifically scaled
specifically to 1, so it would be more correct to say the eigenvector
\(\frac{\vec{x}}{\sum \vec{x}}\)} \(\vec{x}\) that corresponds to the
eigenvalue \(\xi=1\):

\begin{align}
\vec{p} (1) = \mathbf{S} \vec{p} \label{eq:eigenprob}
\end{align}

Solving eigenvectors for large matrices can be very resource intensive and so
this approach isn't suitable for analysing large networks.

Upon iteration \eqref{eq:recurrence} will converge to stable stationary point, as discussed
in \ref{fix}, this approach is known as the power method
\cite{larsonElementaryLinearAlgebra1991a} and is what in practice must be
implemented to solve the stationary distribution of
\eqref{eq:random-surfer-recurrence} and \eqref{eq:recurrence}.


As mentioned in \ref{fix} and \ref{pwalk}, the \emph{Random Surfer} and \emph{Power Walk}
transtition probability matrices are completely dense, that means applying the
power method will not be able to take advantage of using sparse matrix
algorithms.

With some effort however it is possible to express the algorithms in such a way that only involves sparse matrices.

\subsection{Implementing the Models}
\label{sec:orgfb5562c}
To Implement the models, first they'll be implemented using an ordinary matrix and then improved to work with sparse matrices and algorithms, the implementation has been performed with \emph{\textbf{R}} and the preamble is provided in listings \ref{preamble}

\begin{listing}[htbp]
\begin{minted}[]{r}
  if (require("pacman")) {
      library(pacman)
    }else{
      install.packages("pacman")
      library(pacman)
    }

    pacman::p_load(tidyverse, Matrix, igraph, plotly, mise, docstring)
\end{minted}
\caption{\label{preamble}Implemented Packages used in this report}
\end{listing}

\subsection{Implementing the Random Surfer}
\label{sec:org454ff50}
\subsubsection{Small Graph, Ordinary Matrices}
\label{implementing-page-rank-methods}
\paragraph{Example Graph}
\label{example-graph}
Consider the following graph:

\begin{minted}[]{r}
g1 <- igraph::graph.formula(1++2, 1+-8, 1+-5, 2+-5, 2+-7, 2+-8, 2+-6, 2+-9, 3++4, 3+-5, 3+-6, 3+-9, 3+-10, 4+-9, 4+-10, 4+-5, 5+-8, 6+-8, 7+-8)
  plot(g1)
\end{minted}

\begin{center}
\includegraphics[width=.9\linewidth]{media/example-graph-power-walk.png}
\end{center}


\subparagraph{Adjacency Matrix}
\label{adjacency-matrix}
The adjacency Matrix is given by:

\begin{minted}[]{r}
  A <- igraph::get.adjacency(g1, names = TRUE, sparse = FALSE) %>%
    as.matrix()

  ## Adjust the Order
  (A <- A[order(as.integer(row.names(A))), order(as.integer(colnames(A)))])
\end{minted}

\begin{verbatim}
  ##    1 2 3 4 5 6 7 8 9 10
  ## 1  0 1 0 0 0 0 0 0 0  0
  ## 2  1 0 0 0 0 0 0 0 0  0
  ## 3  0 0 0 1 0 0 0 0 0  0
  ## 4  0 0 1 0 0 0 0 0 0  0
  ## 5  1 1 1 1 0 0 0 0 0  0
  ## 6  0 1 1 0 0 0 0 0 0  0
  ## 7  0 1 0 0 0 0 0 0 0  0
  ## 8  1 1 0 0 1 1 1 0 0  0
  ## 9  0 1 1 1 0 0 0 0 0  0
  ## 10 0 0 1 1 0 0 0 0 0  0
\end{verbatim}

\subparagraph{State Distribution}
\label{state-distribution}
The state distribution is the transpose of the adjacency matrix:

\begin{minted}[]{r}
  (p0 <- t(A))
\end{minted}

\begin{verbatim}
  ##    1 2 3 4 5 6 7 8 9 10
  ## 1  0 1 0 0 1 0 0 1 0  0
  ## 2  1 0 0 0 1 1 1 1 1  0
  ## 3  0 0 0 1 1 1 0 0 1  1
  ## 4  0 0 1 0 1 0 0 0 1  1
  ## 5  0 0 0 0 0 0 0 1 0  0
  ## 6  0 0 0 0 0 0 0 1 0  0
  ## 7  0 0 0 0 0 0 0 1 0  0
  ## 8  0 0 0 0 0 0 0 0 0  0
  ## 9  0 0 0 0 0 0 0 0 0  0
  ## 10 0 0 0 0 0 0 0 0 0  0
\end{verbatim}

\subparagraph{Probability Transition Matrix}
\label{probability-transition-matrix}
The probability transition matrix is such that each column of the
initial state distribution (i.e. the transposed adjacency matrix) is
scaled to 1.

\begin{minted}[]{r}
  p0 %*% diag(1/colSums(p0))
\end{minted}

\begin{verbatim}
  ##    [,1] [,2] [,3] [,4] [,5] [,6] [,7] [,8]      [,9] [,10]
  ## 1     0    1    0    0 0.25  0.0    0  0.2 0.0000000   0.0
  ## 2     1    0    0    0 0.25  0.5    1  0.2 0.3333333   0.0
  ## 3     0    0    0    1 0.25  0.5    0  0.0 0.3333333   0.5
  ## 4     0    0    1    0 0.25  0.0    0  0.0 0.3333333   0.5
  ## 5     0    0    0    0 0.00  0.0    0  0.2 0.0000000   0.0
  ## 6     0    0    0    0 0.00  0.0    0  0.2 0.0000000   0.0
  ## 7     0    0    0    0 0.00  0.0    0  0.2 0.0000000   0.0
  ## 8     0    0    0    0 0.00  0.0    0  0.0 0.0000000   0.0
  ## 9     0    0    0    0 0.00  0.0    0  0.0 0.0000000   0.0
  ## 10    0    0    0    0 0.00  0.0    0  0.0 0.0000000   0.0
\end{verbatim}

\begin{enumerate}
\item Create a Function
\label{create-a-function}
\begin{minted}[]{r}
  adj_to_probTrans <- function(adjMat) {
    t(adjMat) %*% diag(1/colSums(t(adjMat)))
  }

  (T <- adj_to_probTrans(A)) %>% round(2)
\end{minted}

\begin{verbatim}
  ##    [,1] [,2] [,3] [,4] [,5] [,6] [,7] [,8] [,9] [,10]
  ## 1     0    1    0    0 0.25  0.0    0  0.2 0.00   0.0
  ## 2     1    0    0    0 0.25  0.5    1  0.2 0.33   0.0
  ## 3     0    0    0    1 0.25  0.5    0  0.0 0.33   0.5
  ## 4     0    0    1    0 0.25  0.0    0  0.0 0.33   0.5
  ## 5     0    0    0    0 0.00  0.0    0  0.2 0.00   0.0
  ## 6     0    0    0    0 0.00  0.0    0  0.2 0.00   0.0
  ## 7     0    0    0    0 0.00  0.0    0  0.2 0.00   0.0
  ## 8     0    0    0    0 0.00  0.0    0  0.0 0.00   0.0
  ## 9     0    0    0    0 0.00  0.0    0  0.0 0.00   0.0
  ## 10    0    0    0    0 0.00  0.0    0  0.0 0.00   0.0
\end{verbatim}
\end{enumerate}

\paragraph{Page Rank Random Surfer}
\label{page-rank-random-surfer}
The random surfer page rank method modifies the probability transition
matrix \(T\) so that the method works also for non-ergodic graphs by
introducing the possibility of a random jump, we'll call the surfer
transition matrix \(S\):

\begin{align}
    S &= \lambda T +  \left( 1- \lambda \right)B :\\
\ \\
    B&= \begin{bmatrix}
    \frac{1}{N} & \frac{1}{N} & \ldots & \frac{1}{N} \\
    \frac{1}{N} & \frac{1}{N} & \ldots & \frac{1}{N} \\
        \vdots      & \vdots      & \ddots & \vdots  \\
    \frac{1}{N} & \frac{1}{N} & \ldots & \frac{1}{N} \\
    \end{bmatrix} \label{eq:bgval1} \\
    N&= \left| \left| V \right| \right| \\
    \lambda &\in [0,1]
\end{align}

\begin{minted}[]{r}
  B <- matrix(rep(1/nrow(T), length.out = nrow(T)**2), nrow = nrow(T))
  l <- 0.8123456789

  S <- l*T+(1-l)*B
\end{minted}

\subparagraph{Eigen Value Method}
\label{eigen-value-method}
The eigenvector corresponding to the the eigenvalue of 1 will be the
stationary point:

\begin{minted}[]{r}
  eigen(S, symmetric = FALSE)
\end{minted}

\begin{verbatim}
eigen() decomposition
$values
 [1]  1.000000e+00 -8.123457e-01 -8.123457e-01  8.123457e-01 -3.407464e-09  3.407464e-09
 [7]  6.878591e-17 -4.393838e-17 -1.126771e-18 -1.292735e-32

$vectors
            [,1]          [,2]          [,3]          [,4]          [,5]          [,6]
 [1,] 0.48726141 -7.071005e-01  1.590774e-03  5.000000e-01  6.735753e-01 -6.735753e-01
 [2,] 0.52676629  7.071005e-01 -1.590774e-03  5.000000e-01  9.622504e-02 -9.622505e-02
 [3,] 0.49149620 -2.975837e-03  7.071050e-01 -5.000000e-01  9.622504e-02 -9.622505e-02
 [4,] 0.48044122  2.975837e-03 -7.071050e-01 -5.000000e-01  2.886751e-01 -2.886751e-01
 [5,] 0.04932738  1.463673e-18 -5.541166e-17  2.124631e-17 -3.849002e-01  3.849002e-01
 [6,] 0.04932738  1.463673e-18  5.541166e-17  2.124631e-17 -3.849002e-01  3.849002e-01
 [7,] 0.04932738  1.463673e-18 -2.077937e-17  2.124631e-17 -3.849002e-01  3.849002e-01
 [8,] 0.04243328 -6.484884e-18 -1.103904e-17  6.319692e-17  8.072508e-09  8.072508e-09
 [9,] 0.04243328  6.952446e-18 -9.740331e-18  6.005334e-17  8.072508e-09  8.072509e-09
[10,] 0.04243328  6.952446e-18 -9.740331e-18  6.005334e-17  8.072508e-09  8.072509e-09
               [,7]          [,8]          [,9]         [,10]
 [1,] -3.963430e-01  3.962600e-01  1.828019e-01 -1.752367e-01
 [2,] -1.291621e-01  2.027302e-01  2.199538e-01 -2.197680e-01
 [3,] -3.955284e-01  3.894308e-02  2.223048e-01 -2.248876e-01
 [4,] -4.215353e-01  1.043870e-01  2.747562e-01 -2.777266e-01
 [5,]  5.166485e-01 -8.109210e-01 -8.798152e-01  8.790721e-01
 [6,]  5.201366e-02 -1.308878e-01 -1.049028e-01  1.056778e-01
 [7,]  1.346275e-01 -1.936007e-01  9.054366e-02 -9.554811e-02
 [8,]  2.547528e-16 -1.352936e-16 -1.025353e-16  1.072771e-16
 [9,]  3.196396e-01  1.965446e-01 -2.821213e-03 -5.466313e-03
[10,]  3.196396e-01  1.965446e-01 -2.821213e-03  1.388344e-02

\end{verbatim}

So in this case the stationary point is

\(\langle -0.49, -0.53, -0.49, -0.48, -0.05, -0.05, -0.05, -0.04, -0.04, -0.04 \rangle\)

which can be verified:

$$
1 \vec{p} = S\vec{p}
$$

\begin{minted}[]{r}
  (p     <- eigen(S)$values[1] * eigen(S)$vectors[,1])
\end{minted}

\begin{verbatim}
  ##  [1] -0.48531271 -0.52732002 -0.49152601 -0.47977477 -0.05288058 -0.05288058
  ##  [7] -0.05288058 -0.04558671 -0.04558671 -0.04558671
\end{verbatim}

\begin{minted}[]{r}
  (p_new <- S %*% p)
\end{minted}

\begin{verbatim}
  ##           [,1]
  ## 1  -0.48531271
  ## 2  -0.52732002
  ## 3  -0.49152601
  ## 4  -0.47977477
  ## 5  -0.05288058
  ## 6  -0.05288058
  ## 7  -0.05288058
  ## 8  -0.04558671
  ## 9  -0.04558671
  ## 10 -0.04558671
\end{verbatim}

However this vector does not sum to 1 so the scale should be adjusted
(for probabilities the vector should sum to 1):

\begin{minted}[]{r}
  (p_new <- p_new/sum(p_new))
\end{minted}

\begin{verbatim}
  ##         [,1]
  ## 1  0.2129185
  ## 2  0.2313481
  ## 3  0.2156444
  ## 4  0.2104889
  ## 5  0.0232000
  ## 6  0.0232000
  ## 7  0.0232000
  ## 8  0.0200000
  ## 9  0.0200000
  ## 10 0.0200000
\end{verbatim}

\subparagraph{Power Value Method}
\label{power-value-method}
Using the power method should give the same result, which it indeed
does, but for the scale:

\begin{minted}[]{r}
  p_new <- p_new *123456789

  while (sum(round(p, 9) != round(p_new, 9))) {
      (p     <- p_new)
      (p_new <- S %*% p)
  }

  p_new
\end{minted}

\begin{verbatim}
  ##        [,1]
  ## 1  26286237
  ## 2  28561500
  ## 3  26622771
  ## 4  25986282
  ## 5   2864198
  ## 6   2864198
  ## 7   2864198
  ## 8   2469136
  ## 9   2469136
  ## 10  2469136
\end{verbatim}

\begin{minted}[]{r}
  p
\end{minted}

\begin{verbatim}
  ##        [,1]
  ## 1  26286237
  ## 2  28561500
  ## 3  26622771
  ## 4  25986282
  ## 5   2864198
  ## 6   2864198
  ## 7   2864198
  ## 8   2469136
  ## 9   2469136
  ## 10  2469136
\end{verbatim}

This answer is however identical in direction, if it scaled to 1 the
same value will be returned:

\begin{minted}[]{r}
  (p_new <- p_new/sum(p_new))
\end{minted}

\begin{verbatim}
  ##         [,1]
  ## 1  0.2129185
  ## 2  0.2313481
  ## 3  0.2156444
  ## 4  0.2104889
  ## 5  0.0232000
  ## 6  0.0232000
  ## 7  0.0232000
  ## 8  0.0200000
  ## 9  0.0200000
  ## 10 0.0200000
\end{verbatim}

\subparagraph{Scaling}
\label{scaling}
However if the initial state sums to 1, then the scale of the stationary
vector will also sum to 1.

\begin{minted}[]{r}
  p     <- c(1, 0, 0, 0, 0, 0, 0, 0, 0, 0)
  p_new <- S %*% p

  while (sum(round(p, 9) != round(p_new, 9))) {
      (p     <- p_new)
      (p_new <- S %*% p)
  }

  cbind(p_new, p)
\end{minted}

\begin{verbatim}
  ##         [,1]      [,2]
  ## 1  0.2129185 0.2129185
  ## 2  0.2313481 0.2313481
  ## 3  0.2156444 0.2156444
  ## 4  0.2104889 0.2104889
  ## 5  0.0232000 0.0232000
  ## 6  0.0232000 0.0232000
  ## 7  0.0232000 0.0232000
  ## 8  0.0200000 0.0200000
  ## 9  0.0200000 0.0200000
  ## 10 0.0200000 0.0200000
\end{verbatim}
\subsubsection{Large Graph, Sparse Matrices using CRS}
\label{sec:orgdf15edd}
\paragraph{Creating the Probability Transition Matrix}
\label{sec:orgb83d023}
Implementing the page rank method on a larger graph requires the use of more efficient form of matrix storage.

An adjacency matrix (atleast in the context of graphs relating to webpages and social networks) will contain elements that are mostly zero because the number of edges leaving any vertex will tend to be significantly less than the total number of vertices.

A matrix exhibiting this property is known as a sparse matrix CITE

The properties of a sparse matrix can be implemented in order to improve performance, one such method to acheive this is \emph{Compressed Sparse Row} (CSR) storage, which involves creating a seperate array of values and corresponding indices. CITE

This is implemented by the Matrix package in \textbf{\emph{R}}. CITE

An sparse matrix can be created using the following syntax, which will return a matrix of the class \texttt{dgCMatrix}:

\begin{minted}[]{r}
library(Matrix)
## Create Example Matrix
n <- 20
m <- 10^6
i <- sample(1:m, size = n); j <- sample(1:m, size = n); x <- rpois(n, lambda = 90)
A <- sparseMatrix(i, j, x = x, dims = c(m, m))

summary(A)
\end{minted}

As before in section \ref{probability-transition-matrix}, the probability transition matrix can be found by:

\begin{enumerate}
\item Transposing the adjacency matrix, then
\item Scaling the columns to one
\end{enumerate}

To implement this for a sparseMatrix of the class \texttt{dgCMatrix}, the same technique of multiplying by a diagonalised matrix may be implemented, however to create this new matrix, a new \texttt{sparseMatrix} will need to be created using the properties of the original matrix, this can be done like so:


\begin{minted}[]{r}
 sparse_diag <- function(mat) {
  #' Diagonal Factors of Sparse Matrix
  #'
  #' Return a Diagonal Matrix of the 1 / colsum() such that
  #' matrix multiplication with this matrix would have all column sums
  #' sum to 1
  #'
  #' This should take the transpose of an adjacency matrix in and the output
  #' can be multiplied by the original matrix to scale it to 1.
  #' i

  ## Get the Dimensions
  n <- nrow(mat)

  ## Make a Diagonal Matrix of Column Sums
  D <- sparseMatrix(i = 1:n, j = 1:n, x = colSums(mat), dims = c(n,n))

  ## Throw away explicit Zeroes
  D <- drop0(D)

  ## Inverse the Values
  D@x <- 1/D@x

  ## Return the Diagonal Matrix
  return(D)
}
D <- sparse_diag(t(A))
summary(D)
\end{minted}

\begin{verbatim}

1000000 x 1000000 sparse Matrix of class "dgCMatrix", with 20 entries
        i      j           x
1   57981  57981 0.011235955
2   61426  61426 0.010309278
3  139900 139900 0.012048193
4  152229 152229 0.009615385
5  175521 175521 0.010204082
6  187782 187782 0.013513514
7  233553 233553 0.011904762
8  288279 288279 0.010309278
9  381442 381442 0.011494253
10 401058 401058 0.014084507
11 541818 541818 0.012820513
12 542888 542888 0.014492754
13 578270 578270 0.010101010
14 595348 595348 0.011764706
15 610432 610432 0.011904762
16 614645 614645 0.012195122
17 776459 776459 0.010989011
18 803589 803589 0.008474576
19 821120 821120 0.009708738
20 821769 821769 0.012658228
\end{verbatim}

and hence the probability transition matrix may be implemented by performing matrix multiplication accordingly:

\begin{minted}[]{r}
summary(t(A) %*% D)
\end{minted}

\begin{verbatim}
1000000 x 1000000 sparse Matrix of class "dgCMatrix", with 20 entries
        i      j x
1  809115  57981 1
2   83355  61426 1
3  649740 139900 1
4  451810 152229 1
5  775788 175521 1
6  364814 187782 1
7  631441 233553 1
8  630438 288279 1
9  681415 381442 1
10 103999 401058 1
11 976802 541818 1
12 217196 542888 1
13 755635 578270 1
14 993420 595348 1
15 206922 610432 1
16 462031 614645 1
17 566334 776459 1
18  66922 803589 1
19 809688 821120 1
20 291405 821769 1
\end{verbatim}

\paragraph{Solving the Random Surfer via the Power Method}
\label{sec:org0e6a159}
Solving the eigenvalues for such a large matrix will not feasible, instead the power method will need to be used to find the stationary point.

However, creating a matrix of background probabilites (denoted by \texttt{B} is section \ref{page-rank-random-surfer}) will not be feasible, it would simply be too large, instead some algebra can be used to reduce \(B\) from a matrix into a vector containing only \(\frac{1-\alpha}{N}\).

The power method is given by:

\begin{align}
\vec{p}= \mathbf{S} \vec{p}
\end{align}

where:

\begin{align}
S &= \alpha \mathbf{T} +  \left( 1 - \alpha \right) \mathbf{B} \\
\vec{p} &= \left( \alpha \mathbf{T} +  \left( 1 - \alpha \right) \mathbf{B} \right) \vec{p}\\
&= \alpha \mathbf{T}\vec{p} +  \left( 1-\alpha \right) \mathbf{B} \vec{p}
\end{align}

Let \(\mathbf{F}= \mathbf{B}\vec{p}\), consider the value of \(\mathbf{F}\) :

\begin{align}
\mathbf{F} &=
\begin{bmatrix}
\frac{1}{N} & \frac{1}{N} & \ldots & \frac{1}{N} \\
\frac{1}{N} & \frac{1}{N} & \ldots & \frac{1}{N} \\
\vdots      & \vdots      & \ddots & \vdots \\
\frac{1}{N} & \frac{1}{N} & \ldots & \frac{1}{N} \\
\end{bmatrix} \label{eq:bgVal2}
\begin{bmatrix}
\vec{p_1} \\ \vec{p_2} \\ \vdots \\ \vec{p_m}
\end{bmatrix}  \\
&= \begin{bmatrix}
\left( \sum^{m}_{i= 0}   \left[ p_i \right]  \right) \times \frac{1}{N} \\
\left( \sum^{m}_{i= 0}   \left[ p_i \right]  \right) \times \frac{1}{N} \\
\vdots  \\
\left( \sum^{m}_{i= 0}   \left[ p_i \right]  \right) \times \frac{1}{N} \\
\end{bmatrix}  \\
& \text{Probabilities sum to 1 and hence:} \\
&= \begin{bmatrix}
\frac{1}{N} \\
\frac{1}{N} \\
\frac{1}{N} \\
\vdots  \\
\frac{1}{N} \\
\end{bmatrix}
\end{align}
So instead the power method can be implemented by performing an algorithm to the effect of:

\begin{minted}[]{r}
## Find Stationary point of random surfer
N     <- nrow(A)
alpha <- 0.8
F     <- rep((1-alpha)/N, nrow(A))  ## A nx1 vector of (1-alpha)/N

## Solve using the power method
p     <- rep(0, length.out = ncol(T)); p[1] <- 1
p_new <- alpha*T %*% p + F

## use a Counter to debug
i <- 0
while (sum(round(p, 9) != round(p_new, 9))) {
    p     <- p_new
    p_new <- alpha*T %*% p + F
    (i <- i+1) %>% print()
}

p %>% head() %>% print()
\end{minted}
\subsection{Power Walk Method}
\label{sec:org6f7c322}
\subsubsection{Introduction}
\label{sec:org0e61de2}

\begin{align}
\mathbf{T} &= \mathbf{B} \mathbf{D}^{-1}_{B} \label{eq:pwalk-def}
\end{align}



where:

\begin{itemize}
\item \(\mathbf{B}= \beta^{\mathbf{A}}\)
\begin{description}
\item[{\(x\beta^{1}\) }] probability of following an edge of weight 1
\item[{\(x\beta^{0}\) }] probability of following an edge of weight 0
\item[{\(x\beta^{-1}\)}] probability of following an edge of weight -
\end{description}
\item \(D_{B} = \mathtt{colsums}(\mathbf{B})\)
\item[{\(\mathbf{A}\)}] The Adjacency Matrix
\end{itemize}

\subsubsection{Ordinary Matrices}
\label{sec:orgb7eb39f}
Solving the Power walk can be done pretty much the same as it is with the random surfer, but doing it with Sparse Matrices is a bit trickier.
\subsubsection{Sparse Matrices}
\label{sec:orgef01898}
\paragraph{Theory; Simplifying Power Walk to be solved with Sparse Matrices}
\label{sec:orgd628789}
The Random Surfer model is:

$$\begin{aligned}
    \mathbf{S} &= \alpha \mathbf{T} +  \mathbf{F}  \label{eq:sparse-RS}\end{aligned}$$

where:

\begin{itemize}
\item \(\mathbf{T}\)

\begin{itemize}
\item is an \(i \times j\) matrix that describes the probability of
travelling from vertex \(j\) to \(i\)

\begin{itemize}
\item This is transpose from the way that \texttt{igraph} produces an adjacency
matrix.
\end{itemize}
\end{itemize}

\item \(\mathbf{F} = \begin{bmatrix} \frac{1}{n} \\ \frac{1}{n} \\ \frac{1}{n} \vdots \end{bmatrix}\)
\end{itemize}

Interpreting the transition probability matrix in this way is such that
\(\mathbf{T}= \mathbf{A}\mathbf{D}^{- 1}_A\) under the following
conditions:


\begin{itemize}
\item No column of \(\mathbf{A}\) sums to zero

\begin{itemize}
\item If this does happen the question arises how to deal with
\(\mathbf{D_\mathbf{A}^{- 1}}\)

\begin{itemize}
\item I've been doing \(\mathbf{D}^{\mathrm{T}}_{\mathbf{A}, i, j} := \mathtt{diag} \left( {\frac{1}{\mathtt{colsums}\left( \mathbf{A} \right)}} \right)\)
and then replacing any \(0\) on the diagonal with 1.
\end{itemize}

\item What is done in the paper is to make another matrix \(\mathbf{Z}\)
that is filled with 0, if a column sum of \(\mathbf{A}\) adds to zero
then that column in \(\mathbf{Z}\) becomes \(\frac{1}{n}\)

\begin{itemize}
\item This has the effect of making each row identical

\item The probability of going from an orphaned vertex to any other
vertex would hence be \(\frac{1}{n}\)

\item The idea with this method is then to use
\(D_\mathbf{\left( A+Z \right)}^{- 1}\) this will be consistent with
the \emph{Random Surfer} the method using \(\mathbf{F}\) in
[[\#eq:sparse-RS][]] \eqref{eq:sparse-RS}
\end{itemize}

where each row is identical that is a 0
\end{itemize}
\end{itemize}

The way to deal with the \emph{Power Walk} is more or less the same.

observe that:

$$\begin{aligned}
   \left( \mathbf{B} = \beta^{\mathbf{A}} \right)\wedge \left( \mathbf{A}_{i, j}\right)\in \mathbb{R}  \implies  \left\lvert \mathbf{B}_{i, j} \right\rvert > 0 \quad \forall i,j>n\in \mathbb{Z}^+ \label{eq:b-is-pos}\end{aligned}$$



Be mindful that the use of exponentiation in
] is not an element wise exponentiation
and not an actual matrix exponential (which would be defined by using
power series and logs but is defined)

So if I have:

\begin{itemize}
\item \(\mathbf{O}_{i, j} := 0, \quad \forall i,j\leq n \in \mathbb{Z}^+\)

\item \(\vec{p_i}\) as the state distribution, being a vector of length \(n\)
\end{itemize}

Then It can be shown (see \eqref{eq:sparse-power-walk}):

$$\begin{aligned}
    \mathbf{O} \mathbf{D}_{\mathbf{B}}^{-1} \vec{p_i} = \mathtt{repeat} (\vec{p} \bullet \vec{\delta^{\tiny \mathrm{T}}} \mathtt{, n}\end{aligned})$$



where:

\begin{itemize}
\item \(\vec{\delta_i} = \frac{1}{\mathtt{colsums} \left( \mathbf{B} \right)}\)
\begin{itemize}
\item A vector\ldots{}(\(n\times 1\) matrix)
\end{itemize}
\item[{\(\vec{1}\) }] is a vector containing all 1's
\begin{itemize}
\item A vector\ldots{}(\(n\times 1\) matrix)
\end{itemize}
\item[{\(\vec{\delta^{\mathrm{T}}}\)}] refers to the transpoxe of \(\vec{\detla}\) (\(1\times n\) matrix)
\item[{\(\vec{\delta^{\mathrm{T}}} \vec{p_{i}}\)}] is some number (because it's a dot product)
\end{itemize}

This means we can do:

\begin{align}
  \overrightarrow{p_{i + 1}} & = \mathbf{T}_{\mathrm{pw}}
  \overrightarrow{p_i}\\
& = \mathbf{BD}_{\mathbf{B}}^{- 1}
  \overrightarrow{p_i}\\
  & = \left( \mathbf{B} - \mathbf{O} + \mathbf{O} \right)
  \mathbf{D}_{\mathbf{B}}^{- 1} \overrightarrow{p_i}\\
  & = \left( \left( \mathbf{B} - \mathbf{O} \right)
  \mathbf{D}_{\mathbf{B}}^{- 1} + \mathbf{OD}_{\mathbf{B}}^{- 1} \right)
  \overrightarrow{p_i}\\
  & = \left( \mathbf{B} - \mathbf{O} \right) \mathbf{D}_{\mathbf{B}}^{- 1}
  \overrightarrow{p_i} + \mathbf{OD}_{\mathbf{B}}^{- 1} \overrightarrow{p_i}\\
  & = \left( \mathbf{B} - \mathbf{O} \right) \mathbf{D}_{\mathbf{B}}^{- 1}
  \overrightarrow{p_i} + \vec{1} (\overrightarrow{\delta^{\mathrm{T}}}
  \overrightarrow{p_i}) \\
  & = \left( \mathbf{B} - \mathbf{O} \right) \mathbf{D}_{\mathbf{B}}^{- 1}
  \overrightarrow{p_i} + \mathtt{rep} (\overrightarrow{\delta^{\mathrm{T}}}
  \overrightarrow{p_i})
\end{align}

where:


Let \((\mathbf{B}-\mathbf{O}) = \mathbf{B_{\mathbf{O}}}\):

\begin{eqnarray*}
  \overrightarrow{p_{i + 1}} & = \mathbf{B_{\mathbf{O}}} \mathbf{D}_{\mathbf{B}}^{- 1}
  \overrightarrow{p_i} + \mathtt{rep} (\overrightarrow{\delta^{\mathrm{T}}}
  \overrightarrow{p_i}) &
\end{eqnarray*}

Now solve \(\tmmathbf{D}_B^{- 1}\) in terms of \(\mathbf{B_{O}}\) :

\begin{align}
  \mathbf{B}_{\mathbf{\mathbf{O}}} = & (\mathbf{B}-\mathbf{O})\\
  \mathbf{B} = & \mathbf{B}_{\mathbf{\mathbf{O}}}
  +\mathbf{O}
\end{align}

If we have \(\delta_{\mathbf{B}}\) as the column sums of\(\tmmathbf{\Beta}\) \(\mathbf{B}\):

\begin{align}
\delta^{-1}_{\mathbf{B}} &= \vec{1}\mathbf{B} \\
&= \vec{1} \left( \mathbf{B_{O}} + \mathbf{O}\right) \\
&= \vec{1}  \mathbf{B_{O}} + \vec{1}\mathbf{O} \\
&= \vec{1} \mathbf{B_{\mathbf{O}}} + \langle n, n, n, ... n \rangle \\
&= \vec{1} \mathbf{B_{\mathbf{O}}} + \vec{1} n \\
\delta_{\mathbf{B}}&=\mathtt{1/(colSums(\mathbf{B_{O}}) + n )}
\end{align}

Then if we have \(\mathit{{\tmstrong{{\tmem{D}}}}}_{\mathit{{\tmem{{\tmstrong{B}}}}}} =
\mathtt{diag} (\delta_{\tmmathbf{B}}) \mathtt{}\):


\[ \begin{array}{lll}
     \mathit{{\tmstrong{{\tmem{D}}}}}_{\mathit{{\tmem{{\tmstrong{B}}}}}}^{- 1}
     & = & \mathrm{diag} \left( \delta^{- 1}_{\mathbf{B}} \right)\\
     & = & \mathtt{diag} \left( \mathtt{ColSums}
     (\mathtt{\tmmathbf{B}_{\tmmathbf{O}}}) + \mathtt{n}
     \right)^{\mathtt{- 1}}
   \end{array} \]

And so the the power method can be implemented using sparse matrices:

\begin{align}
\vec{p_{i+1}} = \mathrm{B_{O}} \enspace \mathrm{diag}\left( \vec{1} \mathbf{B_{O}} + \vec{1}n \right) \vec{p_{i}} + \vec{1} \vec{\delta^{\mathrm{T}}\vec{p_{i}}}
\end{align}

in terms of \textbf{\emph{R}}:

\begin{minted}[]{r}
p_new <- Bo %*% diag(colSums(B)+n) %*% p + rep(t(δ) %*% p, n)

# It would also be possible to sum the element-wise product
(t(δ) %*% p) == sum(δ * p)

# Because R treats vectors the same as a nX1 matrix we could also
# perform the dot product of the two vectors, meaning the following
# would be true in R but not generally

(t(δ) %*% p) == (δ %*% p)
\end{minted}


\subparagraph{Solving the Background Probability}
\label{sec:org3926316}
In this case a vertical single column matrix will represent a vector and \(\otimes\) will represent the outer product (i.e. the \emph{Kronecker Product}):



Define \(\vec{\delta}\) as the column sums of
\[\begin{aligned}
     \vec{\delta} & = \mathtt{colsum} (\text{{\bfseries{B}}})^{- 1}\\
     & = \frac{1}{\overrightarrow{1^{{\scriptsize \ensuremath{\boldsymbol{T}}}}}
     \ensuremath{\boldsymbol{B}}}
   \end{aligned}\]


Then we have:


\[ \begin{aligned}
     \mathbf{OD}_{\mathbf{B}}^{- 1} \overrightarrow{p_i} & = \left(
     \begin{array}{cccc}
       1 & 1 & 1 & \\
       1 & 1 & 1 & \ldots\\
       1 & 1 & 1 & \\
       & \vdots &  & \ddots
     \end{array} \right) \left( \begin{array}{cccc}
       \frac{1}{\delta_1} & 0 & 0 & \\
       0 & \frac{1}{\delta_2} & 0 & \ldots\\
       0 & 0 & \frac{1}{\delta_{13}} & \\
       & \vdots &  & \ddots
     \end{array} \right) \left( \begin{array}{c}
       p_{i, 1}\\
       p_{i, 2}\\
       p_{i, 3}\\
       \vdots
     \end{array} \right) \nonumber \nonumber\\
     & = \left( \begin{array}{cccccc}
       \frac{p_{i, 1}}{\delta 1} & + & \frac{p_{i, 2}}{\delta_2} & + &
       \frac{p_{i, 3}}{\delta_3} & \\
       \frac{p_{i, 1}}{\delta 1} & + & \frac{p_{i, 2}}{\delta_2} & + &
       \frac{p_{i, 3}}{\delta_3} & \ldots\\
       \frac{p_{i, 1}}{\delta 1} & + & \frac{p_{i, 2}}{\delta_2} & + &
       \frac{p_{i, 3}}{\delta_3} & \\
       &  & \vdots &  &  & \ddots
     \end{array} \right) \nonumber \nonumber\\
     & = \left( \begin{array}{c}
       \sum^n_{k = 1} [p_{i, k} \delta_i]\\
       \sum^n_{k = 1} [p_{i, k} \delta_i]\\
       \sum^n_{k = 1} [p_{i, k} \delta_i]\\
       \vdots
     \end{array} \right) \nonumber\\
     & = \left( \begin{array}{c}
       \overrightarrow{\delta^{{\footnotesize \tmmathbf{T}}}}
       \overrightarrow{p_i}\\
       \overrightarrow{\delta^{{\footnotesize \tmmathbf{T}}}} \vec{p}_i\\
       \overrightarrow{\delta^{{\footnotesize \tmmathbf{T}}}} \vec{p}_i\\
       \vdots
     \end{array} \right) \nonumber\\
     & = \overrightarrow{\delta^{{\footnotesize \tmmathbf{T}}}}
     \overrightarrow{p_i} \left( \begin{array}{c}
       1\\
       1\\
       1\\
       \vdots
     \end{array} \right) \nonumber\\
     & = (\overrightarrow{\delta^{{\footnotesize \tmmathbf{T}}}}
     \overrightarrow{p_i})  \vec{1}\\
     & = \mathtt{repeat} (\overrightarrow{\delta} \overrightarrow{p_i}
     \mathtt{, n}) \label{eq:sparse-power-walk}
   \end{aligned} \]
Observe also that If we let \(\vec{\delta}\) and \(p_i\) be 1 dimensional
vectors, this can also be expressed as a dot product:

\begin{center}
\begin{tabular}{ll}
Matrices & Vectors\\
\(\vec{\delta^{\mathrm{T}}} \vec{p_{i}}\) & \(\vec{\delta} \vec{p_{i}}\)\\
\end{tabular}

\end{center}

\paragraph{Practical; Implementing the Power Walk on Sparse Matrices}
\label{sec:org0e119f8}
\subparagraph{Inspect the newly created matrix and create constants}
\label{sec:orge63ca61}
\subparagraph{Setup}
\label{sec:org0447a17}
\begin{enumerate}
\item Load Packages
\label{sec:orge323aa3}
\begin{minted}[]{r}
if (require("pacman")) {
    library(pacman)
  }else{
    install.packages("pacman")
    library(pacman)
  }
  pacman::p_load(Matrix, igraph, plotly, mise, docstring, expm)
  mise()
\end{minted}

\begin{verbatim}
Loading required package: pacman
\end{verbatim}

\item Define function to create DiagonalsSparse Diagonal Function
\label{sec:org5dfeee9}
This doesn't matter for the power walk, real exponents will always give non-zero values anyway
\begin{minted}[]{r}
sparse_diag <- function(mat) {
  #' Diagonal Factors of Sparse Matrix
  #'
  #' Return a Diagonal Matrix containing either 1 / colsum() or 0 such that
  #' matrix multiplication with this matrix would have all columns
  #' sum to 1
  #'
  #' This should take the transpose of an adjacency matrix in and the output
  #' can be multiplied by the original matrix to scale it to 1.
  #' i
  # mat  <- A
  ## Get the Dimensions
  n <- nrow(mat)

  ## Make a Diagonal Matrix of Column Sums
      ## If a column sums to zero the diag can be zero iff the adjacency_matrix>=0
  D <- sparseMatrix(i = 1:n, j = 1:n, x = colSums(mat), dims = c(n,n))

  ## Throw away explicit Zeroes
  D <- drop0(D)

  ## Inverse the Values
  D@x <- 1/D@x

  ## Return the Diagonal Matrix
  return(D)
}
\end{minted}

\item Make an Example Graph
\label{sec:org93f7128}
\begin{minted}[]{r}
g1 <- igraph::erdos.renyi.game(n = 20, 0.2)
A <- igraph::get.adjacency(g1) # Row to column


beta = 0.843234
β = beta
\end{minted}
\item Plot
\label{sec:orgd1e1d18}

\begin{minted}[]{r}
plot(g1)
\end{minted}

\begin{center}
\includegraphics[width=.9\linewidth]{./Media/Example-graph-plot-debug-power-walk.png}
\end{center}
\end{enumerate}

\subparagraph{Power Walk}
\label{sec:org04cadf8}
\begin{enumerate}
\item Define B
\label{sec:org4fc64a5}
\begin{minted}[]{r}
B      <- A
B@x    <- β^(A@x)
B      <- A
B       <- β^A

Bo     <- A

# These two approaches are equivalent
Bo@x   <- β^(A@x) -1   # This in theory would be faster
# Bo     <- β^(A) -1
# Bo     <- drop0(Bo)


  n <- nrow(A)
\end{minted}

\begin{minted}[]{r}
print(B)
\end{minted}

\begin{verbatim}
20 x 20 Matrix of class "dgeMatrix"
          [,1]     [,2]     [,3]     [,4]     [,5]     [,6]     [,7]     [,8]
 [1,] 1.000000 0.843234 1.000000 1.000000 1.000000 0.843234 1.000000 1.000000
 [2,] 0.843234 1.000000 1.000000 1.000000 0.843234 1.000000 1.000000 1.000000
 [3,] 1.000000 1.000000 1.000000 1.000000 1.000000 1.000000 1.000000 1.000000
 [4,] 1.000000 1.000000 1.000000 1.000000 1.000000 1.000000 1.000000 0.843234
 [5,] 1.000000 0.843234 1.000000 1.000000 1.000000 1.000000 1.000000 1.000000
 [6,] 0.843234 1.000000 1.000000 1.000000 1.000000 1.000000 1.000000 0.843234
 [7,] 1.000000 1.000000 1.000000 1.000000 1.000000 1.000000 1.000000 1.000000
 [8,] 1.000000 1.000000 1.000000 0.843234 1.000000 0.843234 1.000000 1.000000
 [9,] 0.843234 1.000000 1.000000 1.000000 0.843234 1.000000 1.000000 1.000000
[10,] 0.843234 0.843234 1.000000 1.000000 1.000000 1.000000 1.000000 1.000000
[11,] 1.000000 1.000000 1.000000 1.000000 1.000000 1.000000 1.000000 1.000000
[12,] 1.000000 1.000000 1.000000 1.000000 1.000000 0.843234 1.000000 0.843234
[13,] 1.000000 1.000000 1.000000 1.000000 1.000000 1.000000 1.000000 0.843234
[14,] 1.000000 0.843234 1.000000 0.843234 1.000000 0.843234 1.000000 1.000000
[15,] 0.843234 1.000000 1.000000 1.000000 1.000000 1.000000 1.000000 1.000000
[16,] 0.843234 1.000000 0.843234 1.000000 1.000000 1.000000 1.000000 1.000000
[17,] 1.000000 1.000000 0.843234 0.843234 1.000000 1.000000 0.843234 0.843234
[18,] 1.000000 1.000000 1.000000 1.000000 0.843234 1.000000 1.000000 1.000000
[19,] 1.000000 0.843234 0.843234 1.000000 1.000000 1.000000 1.000000 0.843234
[20,] 0.843234 1.000000 1.000000 1.000000 1.000000 1.000000 1.000000 1.000000
          [,9]    [,10]    [,11]    [,12]    [,13]    [,14]    [,15]    [,16]
 [1,] 0.843234 0.843234 1.000000 1.000000 1.000000 1.000000 0.843234 0.843234
 [2,] 1.000000 0.843234 1.000000 1.000000 1.000000 0.843234 1.000000 1.000000
 [3,] 1.000000 1.000000 1.000000 1.000000 1.000000 1.000000 1.000000 0.843234
 [4,] 1.000000 1.000000 1.000000 1.000000 1.000000 0.843234 1.000000 1.000000
 [5,] 0.843234 1.000000 1.000000 1.000000 1.000000 1.000000 1.000000 1.000000
 [6,] 1.000000 1.000000 1.000000 0.843234 1.000000 0.843234 1.000000 1.000000
 [7,] 1.000000 1.000000 1.000000 1.000000 1.000000 1.000000 1.000000 1.000000
 [8,] 1.000000 1.000000 1.000000 0.843234 0.843234 1.000000 1.000000 1.000000
 [9,] 1.000000 1.000000 1.000000 1.000000 1.000000 1.000000 1.000000 1.000000
[10,] 1.000000 1.000000 1.000000 1.000000 1.000000 1.000000 0.843234 1.000000
[11,] 1.000000 1.000000 1.000000 1.000000 0.843234 1.000000 0.843234 1.000000
[12,] 1.000000 1.000000 1.000000 1.000000 1.000000 1.000000 1.000000 0.843234
[13,] 1.000000 1.000000 0.843234 1.000000 1.000000 1.000000 1.000000 1.000000
[14,] 1.000000 1.000000 1.000000 1.000000 1.000000 1.000000 1.000000 0.843234
[15,] 1.000000 0.843234 0.843234 1.000000 1.000000 1.000000 1.000000 1.000000
[16,] 1.000000 1.000000 1.000000 0.843234 1.000000 0.843234 1.000000 1.000000
[17,] 1.000000 1.000000 0.843234 0.843234 0.843234 1.000000 1.000000 1.000000
[18,] 1.000000 1.000000 1.000000 1.000000 1.000000 1.000000 1.000000 1.000000
[19,] 1.000000 1.000000 1.000000 1.000000 1.000000 0.843234 1.000000 1.000000
[20,] 1.000000 1.000000 1.000000 1.000000 1.000000 0.843234 1.000000 0.843234
         [,17]    [,18]    [,19]    [,20]
 [1,] 1.000000 1.000000 1.000000 0.843234
 [2,] 1.000000 1.000000 0.843234 1.000000
 [3,] 0.843234 1.000000 0.843234 1.000000
 [4,] 0.843234 1.000000 1.000000 1.000000
 [5,] 1.000000 0.843234 1.000000 1.000000
 [6,] 1.000000 1.000000 1.000000 1.000000
 [7,] 0.843234 1.000000 1.000000 1.000000
 [8,] 0.843234 1.000000 0.843234 1.000000
 [9,] 1.000000 1.000000 1.000000 1.000000
[10,] 1.000000 1.000000 1.000000 1.000000
[11,] 0.843234 1.000000 1.000000 1.000000
[12,] 0.843234 1.000000 1.000000 1.000000
[13,] 0.843234 1.000000 1.000000 1.000000
[14,] 1.000000 1.000000 0.843234 0.843234
[15,] 1.000000 1.000000 1.000000 1.000000
[16,] 1.000000 1.000000 1.000000 0.843234
[17,] 1.000000 0.843234 0.843234 1.000000
[18,] 0.843234 1.000000 0.843234 1.000000
[19,] 0.843234 0.843234 1.000000 1.000000
[20,] 1.000000 1.000000 1.000000 1.000000
\end{verbatim}


\begin{minted}[]{r}
print(Bo)
\end{minted}

\begin{verbatim}
20 x 20 sparse Matrix of class "dgCMatrix"

 [1,]  .        -0.156766  .         .         .        -0.156766  .
 [2,] -0.156766  .         .         .        -0.156766  .         .
 [3,]  .         .         .         .         .         .         .
 [4,]  .         .         .         .         .         .         .
 [5,]  .        -0.156766  .         .         .         .         .
 [6,] -0.156766  .         .         .         .         .         .
 [7,]  .         .         .         .         .         .         .
 [8,]  .         .         .        -0.156766  .        -0.156766  .
 [9,] -0.156766  .         .         .        -0.156766  .         .
[10,] -0.156766 -0.156766  .         .         .         .         .
[11,]  .         .         .         .         .         .         .
[12,]  .         .         .         .         .        -0.156766  .
[13,]  .         .         .         .         .         .         .
[14,]  .        -0.156766  .        -0.156766  .        -0.156766  .
[15,] -0.156766  .         .         .         .         .         .
[16,] -0.156766  .        -0.156766  .         .         .         .
[17,]  .         .        -0.156766 -0.156766  .         .        -0.156766
[18,]  .         .         .         .        -0.156766  .         .
[19,]  .        -0.156766 -0.156766  .         .         .         .
[20,] -0.156766  .         .         .         .         .         .

 [1,]  .        -0.156766 -0.156766  .         .         .         .
 [2,]  .         .        -0.156766  .         .         .        -0.156766
 [3,]  .         .         .         .         .         .         .
 [4,] -0.156766  .         .         .         .         .        -0.156766
 [5,]  .        -0.156766  .         .         .         .         .
 [6,] -0.156766  .         .         .        -0.156766  .        -0.156766
 [7,]  .         .         .         .         .         .         .
 [8,]  .         .         .         .        -0.156766 -0.156766  .
 [9,]  .         .         .         .         .         .         .
[10,]  .         .         .         .         .         .         .
[11,]  .         .         .         .         .        -0.156766  .
[12,] -0.156766  .         .         .         .         .         .
[13,] -0.156766  .         .        -0.156766  .         .         .
[14,]  .         .         .         .         .         .         .
[15,]  .         .        -0.156766 -0.156766  .         .         .
[16,]  .         .         .         .        -0.156766  .        -0.156766
[17,] -0.156766  .         .        -0.156766 -0.156766 -0.156766  .
[18,]  .         .         .         .         .         .         .
[19,] -0.156766  .         .         .         .         .        -0.156766
[20,]  .         .         .         .         .         .        -0.156766

 [1,] -0.156766 -0.156766  .         .         .        -0.156766
 [2,]  .         .         .         .        -0.156766  .
 [3,]  .        -0.156766 -0.156766  .        -0.156766  .
 [4,]  .         .        -0.156766  .         .         .
 [5,]  .         .         .        -0.156766  .         .
 [6,]  .         .         .         .         .         .
 [7,]  .         .        -0.156766  .         .         .
 [8,]  .         .        -0.156766  .        -0.156766  .
 [9,]  .         .         .         .         .         .
[10,] -0.156766  .         .         .         .         .
[11,] -0.156766  .        -0.156766  .         .         .
[12,]  .        -0.156766 -0.156766  .         .         .
[13,]  .         .        -0.156766  .         .         .
[14,]  .        -0.156766  .         .        -0.156766 -0.156766
[15,]  .         .         .         .         .         .
[16,]  .         .         .         .         .        -0.156766
[17,]  .         .         .        -0.156766 -0.156766  .
[18,]  .         .        -0.156766  .        -0.156766  .
[19,]  .         .        -0.156766 -0.156766  .         .
[20,]  .        -0.156766  .         .         .         .
\end{verbatim}

\item Solve the Scaling Matrix
\label{sec:orga7530e2}
We don't need to worry about any terms of \(\delta_{\mathbf{B}} = \mathtt{colsums\left(B\_o\right)+n}\) being 0:

\begin{minted}[]{r}
(δB   <- 1/(colSums(Bo)+n))
\end{minted}

\begin{verbatim}
 [1] 0.05290267 0.05203951 0.05120406 0.05120406 0.05120406 0.05161840
 [7] 0.05039501 0.05246754 0.05079631 0.05120406 0.05120406 0.05161840
[13] 0.05120406 0.05246754 0.05120406 0.05203951 0.05379495 0.05120406
[19] 0.05246754 0.05120406
\end{verbatim}


\begin{minted}[]{r}
(δB   <- 1/(colSums(B)))
\end{minted}

\begin{verbatim}
 [1] 0.05290267 0.05203951 0.05120406 0.05120406 0.05120406 0.05161840
 [7] 0.05039501 0.05246754 0.05079631 0.05120406 0.05120406 0.05161840
[13] 0.05120406 0.05246754 0.05120406 0.05203951 0.05379495 0.05120406
[19] 0.05246754 0.05120406
\end{verbatim}


\item Find the Transition Probability Matrix
\label{sec:org9f015f4}
\begin{minted}[]{r}
  DB   <- diag(δB)
## ** Create the Transition Probability Matrix
## Create the Trans Prob Mat using Power Walk
  T <- Bo %*% DB
\end{minted}

\item Implement the Loop
\label{sec:org501477b}
\begin{minted}[]{r}
## ** Implement the Power Walk
## *** Set Initial Values
  p_new  <- rep(1/n, n)  # Uniform
  p      <- rep(0, n)    # Zero
  η      <- 10^(-6)
## *** Implement the Loop

 while (sum(abs(p_new - p)) > η) {
    (p <- as.vector(p_new)) # P should remain a vector
    sum(p <- as.vector(p_new)) # P should remain a vector
     p_new  <- T %*% p + rep(t(δB) %*% p, n)
  }
## ** Report the Values
print(paste("The stationary point is"))
print(p)
\end{minted}

\begin{verbatim}
[1] "The stationary point is"
 [1] 0.04882572 0.04963556 0.05044542 0.05044541 0.05044543 0.05004049
 [7] 0.05125527 0.04923064 0.05085035 0.05044543 0.05044542 0.05004049
[13] 0.05044542 0.04923064 0.05044543 0.04963557 0.04801586 0.05044542
[19] 0.04923063 0.05044542
\end{verbatim}
\end{enumerate}

\subsection{Creating a Package}
\label{sec:orga855ef6}
\section{Relating the Power Walk to the Random Surfer}
\label{relating-terms-in-power-walk-to-random-surfer}
\subsection{Introduction}
\label{sec:org74965c9}
These are notes relating to \cite[\textsection 3.3]{parkPowerWalkRevisiting2013}

So if a term in the Power Walk can be related to \(\alpha\) in the random
surfer, which is in turn \(\xi_2\), I'll be able to understand it better. \footnote{Although I'm not quite sure why \(\alpha\) is \(\xi_{2}\) either}

Consider the equation:


\begin{align*}
\mathbf{T}&= \mathbf{B}\mathbf{D}_{\mathbf{B}}^{- 1} \\
&= \left( \mathbf{B}+  \mathbf{O} - \mathbf{O} \right) \mathbf{D}_{\mathbf{B}}^{- 1} \\
\end{align*}


Break this into to terms so that we can simplify it a bit:


\begin{align*}
    \mathbf{T} &= \Bigg[ \left( \mathbf{B}- \mathbf{O} \right)\mathbf{D}_{\mathbf{B}}^{- 1} \Bigg] + \Bigg\{  \mathbf{O}\mathbf{D}_{\mathbf{B}}^{- 1} \Bigg\}
\end{align*}
\subsection{Value of [1st Term]}
\label{value-of-1st-term}
Observe that for all \(\forall i,j\in \mathbb{Z}^+\):


\begin{align*}
\mathbf{A}_{i, j} \in \left\{0, 1\right\} \\
\implies  \mathbf{B}^{\mathbf{A}_{i, j}} &\in \left\{\beta^0, \beta^1\right\} \\
                     &= \left\{1, \beta \right\}  \\
                      \implies  \beta \mathbf{A} = \left\{1, \beta \right\}
\end{align*}


Using this property we get the following


\begin{align*}
\mathbf{B}_{i,j}- \mathbf{O}_{i,j} = \left( \beta^{\mathbf{A}_{i,j}} -1 \right) &=
\begin{cases}
    0      , &\enspace \mathbf{A}_{i,j}=0  \\
    \beta-1, &\enspace \mathbf{A}_{i,j}=1  \\
\end{cases} \\
\left( \beta- 1 \right) \mathbf{A}_{i,j} &=
\begin{cases}
    0      , &\enspace \mathbf{A}_{i,j}=0  \\
    \beta-1, &\enspace \mathbf{A}_{i,j}=1  \\
\end{cases} \\
\end{align*}


This means we have


\begin{align*}
\mathbf{A} \in \left\{0, 1\right\} \forall i,j  \implies   \mathbf{B}_{i,j}- \mathbf{O}_{i,j} &= \left( \beta-1 \right) \mathbf{A}_{i,j}
\end{align*}



\begin{align*}
\mathbf{B}&= \left( \mathbf{B}+  \mathbf{O}- \mathbf{O} \right) \\
&= \left( \mathbf{B}- 1 \right)
\end{align*}

\subsection{Value of \{2nd Term\}}
\label{value-of-2nd-term}
\begin{align*}
\mathbf{O} \mathbf{D_B^{- 1}} &=
\begin{pmatrix}
    1 & 1      & 1 &        \\
    1 & 1      & 1 &\cdots  \\
    1 & 1      & 1 &        \\
      & \vdots &   &\ddots
\end{pmatrix}
\begin{pmatrix}
    \frac{1}{\delta_1} & 1                    & 1                   & \\
    1                  & \frac{1}{\delta_{2}} & 1 \cdots            & \\
    1                  & 1                    &  \frac{1}{\delta_3} & \\
               & \vdots &             &                     \ddots
\end{pmatrix}
\\
&= n
\begin{pmatrix}
    \frac{1}{n} & \frac{1}{n}      & \frac{1}{n} &        \\
    \frac{1}{n} & \frac{1}{n}      & \frac{1}{n} &\cdots  \\
    \frac{1}{n} & \frac{1}{n}      & \frac{1}{n} &        \\
      & \vdots &   &\ddots
\end{pmatrix}
\begin{pmatrix}
    \frac{1}{\delta_1} & 1                    & 1                   &        \\
    1                  & \frac{1}{\delta_2}    & 1                   & \cdots \\
    1                  & 1                    &  \frac{1}{\delta_3} &        \\
                       & \vdots               &                     & \ddots
\end{pmatrix}
\\
&= n \mathbf{E}\mathbf{D_B}^{-1}
\end{align*}


where the following definitions hold (\(\forall i, j \in \mathbb{Z}^+\)):

\begin{itemize}
\item \(\mathbf{E}_{i, j} = \frac{1}{n}\)
\item \(\mathbf{D_B}^{-1}_{k, k} = \frac{1}{\delta_k}\)
\item The value of \(\delta\) is value that each term in a column must be
divided by to become zero, in the case of the power walk that is just
\(\frac{1}{\mathtt{colSums}\left( \mathbf{B} \right)} = \vec{1}\mathbf{B}\),
but if there were zeros in a column, it would be necessary to swap out
the \$0\$s for \$1\$s and then sum in order to prevent a division by zero
issue and because the 0s should be left.
\item \(\mathbf{A}\in \left\{0, 1\right\} \forall i,j\) is the unweighted
adjacency matrix of the relevant graph.
\end{itemize}

putting this all together we can do the following:


\begin{align*}
\mathbf{T}&= \mathbf{B}\mathbf{D}^{- 1}_{\mathbf{B}} \\
&= \left( \mathbf{B}+  \mathbf{O} - \mathbf{O} \right) \mathbf{D}_{\mathbf{B}}^{- 1} \\
&= \left( \mathbf{B}- \mathbf{O} \right)\mathbf{D}_{B}^{- 1}  +  \mathbf{O} {\mathbf{D}_{\mathbf{B}}^{- 1}} \\
 \intertext{From above:} \\
&= \left( \beta- 1 \right) \mathbf{A}_{i,j} +  n \mathbf{E} \mathbf{D}_{\mathbf{B}}^{- 1}\\
&= \mathbf{A}_{i,j}\left( \beta- 1 \right)  +  n \mathbf{E} \mathbf{D}_{\mathbf{B}}^{- 1}\\
 \intertext{because $\mathbf{D} \mathbf{D}^{- 1} = \mathbf{I}$ we can multiply one side through:} \\
&= \mathbf{D}_{\mathbf{A}} \mathbf{D}_{\mathbf{A}}^{- 1}\mathbf{A}_{i,j}\left( \beta- 1 \right)  +  n \mathbf{E} \mathbf{D}_{\mathbf{B}}^{- 1}\\
\end{align*}


But the next step requires showing that:


\begin{align*}
\left( \beta-1 \right)\mathbf{D}_\mathbf{A} \mathbf{D}_{\mathbf{B}}^{- 1} &= \mathbf{I} - n \mathbf{D}_{B}^{- 1}
\end{align*}

\subsection{Equate the Power Walk to the Random Surfer}
\label{sec:orgce51980}
Define the matrix \(\mathbf{D}_{\mathbf{M}}\):

\begin{align}
    \mathbf{D}_{\mathbf{M}} = \mathrm{diag}\left( \mathtt{colSum} \left( \mathbf{M} \right) \right) &= \mathrm{diag} \left( \vec{1} \mathbf{M} \right)
\end{align}


To scale each column of that matrix to 1, each column will need to be divieded by the column sum, unless the column is already zero, this needs to be done to turn an adjacency matrix into a matrix of probabilities:

\begin{align}
    \mathbf{D}_{\mathbf{A}} ^{- 1} :  \left[     \mathbf{D}_{\mathbf{A}} ^{- 1}  \right]_i =
    \begin{cases}
	0 ,& \quad \left[ \mathbf{D}_{\mathbf{A}} \right]_i = 0 \\
	\left[ \frac{1}{\mathbf{D}_{\mathbf{A}}} \right] ,& \enspace \enspace \left[ \mathbf{D}_{\mathbf{A}} \right]_i \neq 0
    \end{cases}
\end{align}

In the case of the power walk \(\mathbf{B}= \beta^{\mathbf{A}} \neq 0\) so it is sufficient:

\begin{align}
    \mathbf{D}_{\mathbf{B}}^{- 1} &= \frac{1}{\mathrm{diag}\left( \vec{1} \left(\mathbf{\beta^{\mathbf{A}}  \right) } \right)}
\end{align}


Recall that the \emph{power walk} gives a transition probability matrix:

\begin{align}
%    \mathbf{T} &= \mathbf{a} \text{\fboxsep=.2em\fbox{$x$}} \\
    \text{\textbf{Power Walk}} \nonumber \\
\mathbf{T} &= \text{\fboxsep=.2em\fbox{$\mathbf{A}\mathbf{D}_{\mathbf{A}}^{- 1}$}}  \mathbf{D}_{\mathbf{A}} \left( \beta - 1 \right) \mathbf{D}_{\mathbf{B}}^{- 1} + \text{\fboxsep=.2em\fbox{$\mathbf{E}$}} n \mathbf{D}_{\mathbf{B}}^{- 1}  \label{eq:pwbx}\\
    \text{\textbf{Random Surfer}} \nonumber \\
    \mathbf{T} &= \alpha \text{\fboxsep=.2em\fbox{$\mathbf{A}\mathbf{D}_{\mathbf{A}}^{- 1}$}}  + \left( 1-\alpha \right) \text{\fboxsep=.2em\fbox{$\mathbf{E}$}}
\end{align}

So these are equivalent when:

\begin{align}
\mathbf{D}_{\mathbf{A}}   \left( \beta -  1 \right)\mathbf{D}_{\mathbf{B}^{- 1}} &=\mathbf{I}  \alpha \label{fl} \\
    \ \nonumber \\
  \vec{1}  \left( 1- \alpha \right) &=  - n \mathbf{D}_{\mathbf{B}}^{- 1}  \nonumber \\
    \implies  \vec{1}\alpha &=  \vec{1}- n \mathbf{D}_{\mathbf{B}}^{- 1} \label{st} \\
    \intertext{Hence we have:} \notag \\
\mathbf{D}_{\mathbf{A}}  \left( \beta -  1 \right)\mathbf{D}_{\mathbf{B}}^{- 1} &=  \vec{1}\alpha =  \mathbf{I}- n \mathbf{D}_{\mathbf{B}}^{- 1} \label{eq:eqalpha}
\end{align}


Solving for \(\beta\)  with \eqref{fl} :

\begin{align}
    \beta&= \frac{1- \Theta}{\Theta}\\
%    \beta&= \frac{\alpha - \mathbf{D}_{\mathbf{A}}\mathbf{D}_{\mathbf{B}}^{- 1}}{\mathbf{D}_{\mathbf{A}}\mathbf{D}_{\mathbf{B}}^{-1}}
\end{align}

where: \footnote{NOTE: Similar to a signmoid function, which is a solution to \(p \propto p(1-p)\), I wonder if this provides a connection to the exponential nature of the power walk
`erdos.renyi`erdos.renyi``}

\begin{itemize}
\item \(\Theta = \mathbf{D}_{\mathbf{A}} \mathbf{D}_{\mathbf{B}}^{- 1}\)
\end{itemize}

but we can't really do this so instead:

\[
\beta \mathbf{1}_{\tiny \left[ n,n \right]}  = \left( 1 - \Theta \right) \Theta^{-1} \label{eq:betadef}
\]

If \(\beta\) is set accordingly then by \eqref{eq:eqalpha}:

\begin{align}
    \mathbf{A}\left( \beta- 1 \right) \mathbf{D}_{\mathbf{B}}^{- 1} &= \alpha = \mathbf{I}- n \mathbf{D}_{\mathbf{B}}^{- 1} \nonumber \\
     \implies  \mathbf{A}\left( \beta- 1 \right) \mathbf{D}_{\mathbf{B}}^{- 1} &=  \mathbf{I}- n \mathbf{D}_{\mathbf{B}}^{- 1}
\end{align}

And setting \(\Gamma = \mathbf{I}- n \mathbf{D}_{\mathbf{B}}^{- 1}\)  from \eqref{st} and putting in \eqref{eq:pwbx} we have:

\begin{align}
\mathbf{T} &= \text{\fboxsep=.2em\fbox{$\mathbf{A}\mathbf{D}_{\mathbf{A}}^{- 1}$}}  \mathbf{D}_{\mathbf{A}} \left( \beta - 1 \right) \mathbf{D}_{\mathbf{B}}^{- 1} + \text{\fboxsep=.2em\fbox{$\mathbf{E}$}} n \mathbf{D}_{\mathbf{B}}^{- 1}  \nonumber \\
  \mathbf{T} &= \Gamma \text{\fboxsep=.2em\fbox{$\mathbf{A}\mathbf{D}_{\mathbf{A}}^{- 1}$}}  + \left( 1-\Gamma \right) \text{\fboxsep=.2em\fbox{$\mathbf{E}$}} \nonumber \\
  \ \nonumber \\
  \mathbf{T} &= \Gamma \mathbf{A}\mathbf{D}_{\mathbf{A}}^{- 1}  + \left( 1-\Gamma \right) \mathbf{E}
  \end{align}

Where \(\mathbf{E}\) is square matrix of \(\frac{1}{n}\) as in \eqref{eq:bgval1}  \eqref{eq:bgVal2}

\subsection{Conclusion}
\label{sec:org93d9ea8}
So when the adjacency matrix is stictly boolean, the power walk is equivalent to the random surfer.

\subsection{The Second Eigenvalue}
\label{sec:org6281be1}
\subsubsection{The Random Surfer}
\label{sec:orgfc8fc48}
The Second eigenvalue \(\xi_2\) of the Power Surfer is less than \(\alpha\) (\href{Proposal/Propsal.org}{See 3.2; Stability and Concvergence, of proposal}).
\subsubsection{Power Walk}
\label{sec:org79edf26}
Because the Power Walk relates to the random surfer as demonstrated in section \ref{relating-terms-in-power-walk-to-random-surfer}, what can be said about \(\xi_{2}\)
\paragraph{Applying this to Power Walk}
\label{sec:org6eda0f1}
Let \(\Lambda_{\left( 2 \right)}\left( \mathbf{T} \right) = \lambda_2\) return the second value of a transition, probability Matrix, then observe that:


\begin{align}
    \Lambda_{\left( 2 \right)} \left( \mathbf{T}_{\text{\tiny RS}} \right)  \leq \left\lvert \alpha \right\rvert  \implies      \Lambda_{\left( 2 \right)} \left( \mathbf{T}_{\text{\tiny PW}} \right) \leq \left\lvert \frac{\alpha - \mathbf{D}_{\mathbf{a}} \mathbf{D}_{\mathbf{B}}^{- 1}}{\mathbf{D}_{\mathbf{A}}\mathbf{D}_{\mathbf{B}}^{-1}}  \right\rvert
\end{align}

where:


\begin{itemize}
\item \(\lambda_{\left( 2 \right)} \left( \mathbf{T} \right)\) refers to the transition probability matrix of the power walk and random surfer approaces as indicated.
\end{itemize}
\subparagraph{My attempt}
\label{sec:org52fc0ae}
\begin{align}
    \beta \mathbf{1}_{\tiny \left[ n, n \right] }    &= \frac{1- \Theta}{\Theta} \label{eq:betasig}\\
%    \beta&= \frac{\alpha - \mathbf{D}_{\mathbf{A}}\mathbf{D}_{\mathbf{B}}^{- 1}}{\mathbf{D}_{\mathbf{A}}\mathbf{D}_{\mathbf{B}}^{-1}}
\end{align}

where:
\begin{itemize}
\item \(\Theta = \mathbf{D}_{\mathbf{A}} \mathbf{D}_{\mathbf{B}}^{- 1}\)
\end{itemize}

So I thought maybe if I could find a value of \(\beta\) that satisfied \eqref{eq:betasig} then I could show circumstances under which \(\left\lvert \xi_2 \right\rvert < \alpha\).

Seemingly it's only satisfied where \(\beta = 1\) though, using this simulation:

\begin{minted}[]{r}
g1 <- igraph::erdos.renyi.game(n = 9, 0.2)
A <- igraph::get.adjacency(g1) # Row to column
A <- t(A)
# plot(g1)

## * Finding beta values to behave like Random Surfer
  beta <- 10
  B <- beta^A

  DA     <- PageRank::create_sparse_diag_sc_inv_mat(A)
  DB_inv <- PageRank::create_sparse_diag_scaling_mat(B)

 THETA <- DA %*% DB_inv

THETA <- function(A, beta) {
  B  <- beta^A
  DA     <- PageRank::create_sparse_diag_sc_inv_mat(A)
  DB_inv <- PageRank::create_sparse_diag_scaling_mat(B)
  return(DA %*% DB_inv)
}

THETA_inv <- function(A, beta) {
  B  <- beta^A
  DB     <- PageRank::create_sparse_diag_sc_inv_mat(B)
  DA_inv <- PageRank::create_sparse_diag_scaling_mat(A)
  return(DA %*% DB_inv)
}

beta_func <- function(A, beta) {
    return(1-THETA(A, beta^A) %*% THETA_inv(A, beta^A))
}

THETA(A, 10) %*% THETA_inv(A, 10)


eta <- 10^-6
beta <- 1.01
while (mean(beta*matrix(1, nrow(A), ncol(A)) - beta_func(A, beta)) > eta) {
    beta <- beta + 0.01
    print(beta)
    print(diag(beta_func(A, beta)))
    print(beta*matrix(1, nrow(A), ncol(A)))
    print(beta_func(A, beta))
#    Sys.sleep(0.1)
}

beta


diag(beta_func(A, beta))
beta


## * blah
\end{minted}
\section{Simulating the Structure of the Web}
\label{sec:org37b8043}
A graph of the internet is \emph{scale free}, this means that the number of nodes of a graph (\(n\)), having \(j\) edges is given by \cite[\textsection 10.7.2]{langvilleGooglePageRankScience2012}:

\begin{align}
n \propto j^{-k}, \quad \exists k \in \mathbb{R}
\end{align}

The \emph{Erdos Renyi} game is a random network, a superior approach to model the web is to use a scale free networks \cite{barabasiPhysicsWeb2001} such as the Barabasi-Albert graph \cite{barabasiScalefreeCharacteristicsRandom2000}

\section{Investigating the Second EigenValue}
\label{sec:org3b1b856}

Maybe I should look at the most appropriate way to simulate social network links, one possibility is \href{https://crpit.scem.westernsydney.edu.au/confpapers/CRPITV144Zeng.pdf}{this paper } \cite{zengPracticalSimulationMethod2013}.

Actually there is a data set available
 \cite{garritanoWikipediaArticleNetworks2019}, I should just analyse that, see \href{file:///home/ryan/Dropbox/DataSci/Visual\_Analytics/Assessment/the-marvel-universe-social-network/plotly3d\_Marvel.r}{how
it was done in Visual Analytics as a reminder}.

Using the Wikipedia ArtYeah I think that's right, thaicle compare density and Determinant.

Is the determinant easily calculated for a large matrix?
  It appears to diverge

Will the determinant diverge for large matrices?
Will the prob of making edges in the game just be the density?

Look at comparing the determinant and the density of the wikipedia adjacency matrix.

What are some ways that we can model the second eigenvalue?

\subsection{Plotting Various Values}
\label{sec:orgc863a2e}

There is some relationship between the determinant and the density, check out the pairs plot:



\begin{minted}[]{r}
  library(pacman)
  pacman::p_load(PageRank, devtools, Matrix, igraph, tidyverse)
n <- 20
p <- 1:n/n
beta <- 1:n/n
beta <- runif(n)*100
sz <- 1:n/n+10
input_var <- expand.grid("n" = n, "p" = p, "beta" = beta, "size" = sz)
input_var


random_graph <- function(n, p, beta, size) {
      g1 <- igraph::erdos.renyi.game(n = sz, p)
      A <- igraph::get.adjacency(g1) # Row to column
      A <- Matrix::t(A)

      A_dens <- mean(A)
      T      <- PageRank::power_walk_prob_trans(A)
      e2     <- eigen(T, only.values = TRUE)$values[2] # R orders by descending magnitude
      A_det  <- det(A)
      return(c(abs(e2), A_det))
}

## TODO this should use pmap.
Y <- matrix(ncol = 2, nrow = nrow(input_var))
for (i in 1:nrow(input_var)) {
  X <- as.vector(input_var[i,])
  Y[i,] <-  random_graph(X$n, X$p, X$beta, X$size)
}
if (sum(abs(Y) != abs(Re(Y))) == 0) {
  Y <- Re(Y)
}
nrow(input_var)
nrow(Y)
Y <- as.data.frame(Y); colnames(Y) <- c("eigenvalue2", "determinant")

data <- cbind(input_var, Y)

ggplot(data, aes(x = determinant, y = eigenvalue2, size = beta, color = size, shape = factor(n))) +
  geom_point() +
  labs(x = "Determinant of Adjacency Matrix", y = "Second Eigenvalue of Power Walk Transition Probability Matrix") +
  scale_size_continuous(range = c(0.1,1))
\end{minted}

\begin{center}
\includegraphics[width=.9\linewidth]{../Media/EigenValue_Determinant.png}
\end{center}


\begin{minted}[]{r}
  library(pacman)
  pacman::p_load(PageRank, devtools, Matrix, igraph, tidyverse)
n <- 100
p <- 1:n/n
beta <- 1:n/n
beta <- runif(n)*100
sz <- 1:n/n+10
input_var <- expand.grid("n" = n, "p" = p, "beta" = beta, "size" = sz)
input_var


random_graph <- function(n, p, beta, size) {
      g1 <- igraph::erdos.renyi.game(n = sz, p)
      A <- igraph::get.adjacency(g1) # Row to column
      A <- Matrix::t(A)

      A_dens <- mean(A)
      T      <- PageRank::power_walk_prob_trans(A)
      e2     <- eigen(T, only.values = TRUE)$values[2] # R orders by descending magnitude
      A_det  <- det(A)
      return(c(abs(e2), A_dens))
}

## TODO this should use pmap.
Y <- matrix(ncol = 2, nrow = nrow(input_var))
for (i in 1:nrow(input_var)) {
  X <- as.vector(input_var[i,])
  Y[i,] <-  random_graph(X$n, X$p, X$beta, X$size)
}
if (sum(abs(Y) != abs(Re(Y))) == 0) {
  Y <- Re(Y)
}
nrow(input_var)
nrow(Y)
Y <- as.data.frame(Y); colnames(Y) <- c("eigenvalue2", "determinant")

data <- cbind(input_var, Y)

ggplot(data, aes(x = determinant, y = eigenvalue2, color = size, shape = factor(n))) +
  geom_point(base_size = 99, aes(size = beta)) +
  labs(x = "Density of Adjacency Matrix", y = "Second Eigenvalue of Power Walk Transition Probability Matrix") +
  scale_size_continuous(range = c(0.1,1))
\end{minted}

Maybe this looks like a Chi distribution?

\begin{minted}[]{r}
chival <- dchisq(seq(from = 0, to = 40, length.out = 100), df = 10)*6
index  <- seq(from = 0, to = 2, length.out = 100)
chidata  <- data.frame(index = index, chi = chival)
ggplot(data) +
  geom_point(mapping = aes(x = determinant, y = eigenvalue2, size = beta, color = size, shape = factor(n))) +
  geom_line(data = chidata, mapping = aes(x = index, y = chi)) +
  scale_size_continuous(range = c(0.1,1)) +
  labs(x = "Density of Adjacency Matrix", y = "Second Eigenvalue of Power Walk Transition Probability Matrix")
\end{minted}

\begin{center}
\includegraphics[width=.9\linewidth]{../Media/EigenValue_Density_Chi.png}
\end{center}

\subsection{Model the log transformed data using a linear regression or log(-x) regression}
\label{sec:orga7d2a24}

\begin{align}
    \xi_2 &= \left( 1-  \frac{\sum^{n}_{i= 1} \sum^{n}_{j= 1}   \mathbf{A}_{i,j}  }{n^{2}} \right)^{0.6} \cdot  e^{- 0.48} \pm \Delta
\end{align}

\subsubsection{Change the colour of each model by using pivot\textsubscript{longer}}
\label{sec:org60eb1ea}
\subsection{Could I get better performance by also considering the determinant?}
\label{sec:org272e5aa}
No not really, it terms of accuracy

\subsection{Is the determinant faster or slower?}
\label{sec:org8edfd48}
Significantly slower for large matrices.
\subsection{Import wikipedia data}
\label{sec:org6df90c9}
\begin{itemize}
\item \sout{Import the wikipedia data}
\item \sout{Measure the density}
\item \sout{Use the density to guess the \(p\) of the game}
\begin{itemize}
\item \sout{Justify the witht the scatterplot matrix}
\end{itemize}
\item \sout{Measure the affect of different \(\beta\) values on \(\lambda_2\) for graphs ov various sizes given that \(p\) value.}
\begin{itemize}
\item \sout{Or atleast a range within that prob}

use a \emph{Barabassi-Albert} Random Graph through the \textasciitilde{}igraph::
\end{itemize}
\end{itemize}

\section{Cauchy Integral Formula}
\label{cauchy-integral-formula}
This is from section 54 of the book, isn't it nice that it more or less
just works hey? \cite{zhangMakingEigenvectorBasedReputation2004}

\begin{align}
f\left( a \right) \frac{1}{2\pi i} \oint \frac{f\left( z \right)}{z- a}\mathrm{d}z
\end{align}

In view of this equation then: \cite{zhangMakingEigenvectorBasedReputation2004}

$$\begin{aligned}
\left| \int_C \frac{f\left( z \right)}{z- z_0} \mathrm{d}z - 2 \pi i f\left( z_0 \right) \right|<2 \pi \varepsilon
\end{aligned}$$

Some Images: \cite{ngStableAlgorithmsLink2001}

\begin{figure}[htbp]
\centering
\includegraphics[width=12cm]{media/my-self-rep-frac.png}
\caption{\label{testim}This image is for testing purposes \cite{moskowitzLibraryGuidesWikipedia}}
\end{figure}

\begin{figure}[htbp]
\centering
\includegraphics[width=12cm]{media/tikz/Snowflake.png}
\caption{\label{testtikzins}This is a tikz image inserted as a png from imagemagick}
\end{figure}


@@latex: \includestandalone[]{./media/tikz/Snowflake}



\subsection{Heading 2}
\label{heading-2}
\subsubsection{Heading 3}
\label{heading-3}
\begin{minted}[]{sh}
  echo "Hello World"
\end{minted}


\paragraph{Heading 4}
\label{heading-4}
\subparagraph{Heading 5}
\label{heading-5}
\begin{enumerate}
\item Heading 6
\label{heading-6}
Arbitrary Code:

\begin{minted}[]{sh}
  n/bash

  # Print Help
  if [ "$1" == "-h" ]; then
      echo "Usage: `basename $0` <Format> <CSS>"
      style=~/Dropbox/profiles/Emacs/org-css/github-org.css
      exit 0
  fi

  # Make a working File from clipboard
  filename=lkjdskjjalkjkj392jlkj
  xclip -o -selection clipboard >> $filename
  LocalFile=$filename.org

  pandoc -s  -f org -t gfm $filename -o $filename

  echo "
  This was converted from `org` to `md` using `pandoc -t gfm` at time:
  $(date --utc +%FT%H-%M-%S)
  " >> $filename

  cat $filename | xclip -selection clipboard
  rm $filename

  nv & disown
  echo "Conversion from Org Successful, MD is in Clipboard"

  exit 0
\end{minted}
\end{enumerate}
\section{Appendix}
\label{sec:orgd22d506}

\begin{listing}[htbp]
\begin{minted}[]{r}
library(Matrix)
library(igraph)
n <- 200
m <- 5
power <- 1
g <- igraph::sample_pa(n = n, power = power, m = m, directed = FALSE)
plot(g)
A <- t(get.adjacency(g))
plot(A)
image(A)


# Create a Plotting Region
par(pty = "s", mai = c(0.1, 0.1, 0.4, 0.1))


# create the image

title=paste0("Undirected Barabassi Albert Graph with parameters:\n Power = ", power, "; size = ", n, "; Edges/step = ", round(m))
image(A, axes = FALSE, frame.plot = TRUE, main = title, xlab = "", ylab = "",  )
\end{minted}
\caption{\label{r-den_undir_ba}\textbf{\emph{R}} code to produce an image illustrating the density of a simulated Barabasi-Albert graph, the \emph{Barabasi-Albert} graph is a good analouge for the link structure of the internet \cite{langvilleGooglePageRankScience2012,barabasiPhysicsWeb2001,barabasiScalefreeCharacteristicsRandom2000} see the output in figure \ref{fig:den_undir_ba}}
\end{listing}

\begin{figure}[htbp]
\centering
\includegraphics[width=12cm]{media/DensityUndirectedBA.png}
\caption{\label{fig:den_undir_ba}Plot of the adjacency matrix corresponding to a Barabassi-Albert (i.e. \emph{Scale Free}) Graph produced by listing \ref{r-den_undir_ba}, observe the matrix is quite sparse.}
\end{figure}
\section{Graph Diagrams}
\label{sec:orgf487c2a}
Graph Diagrams shown in \ref{markov} where produced using \texttt{DOT} (see \cite{DOTLanguage,DOTGraphDescription2020}).
\section{Extensions to this Report}
\label{sec:orgf6a471e}
Accellerating the Computatoin of Page Rank \cite{langvilleGooglePageRankScience2012}

\section{Is the power Walk transition prob matrix a stochastic because it may contain negatives?}
\label{sec:orgd7d83fe}
\section{Look at the Trace of the Matrix as a comparison point}
\label{sec:org2603cfa}
\section{TODO Diamater}
\label{sec:org027098e}
Diamater of the web sounds like a fun read \cite{albertDiameterWorldWideWeb1999}
\section{Improving the Performance of Page Rank}
\label{sec:org14544c9}

This:

\begin{quote}
Another approach involves involves reordering the problem and taking advantage
of the fact that the transition probability matrix is sparse  in order
to produce a new algorithm which cannot perform worse than the \emph{power method}
but has been shown to improve the rate of convergence in certain cases.
\cite{langvilleReorderingPageRankProblem2006}.
\end{quote}


There was also a book that I downloaded that mentioned it
\end{document}